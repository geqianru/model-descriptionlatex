\documentclass[preprint,12pt]{elsarticle}
\usepackage{amsthm}
\usepackage[english]{babel}
\usepackage{longtable}
\usepackage{amssymb}
\usepackage{amsmath}
\usepackage{hyperref}
\usepackage{latexsym}
\usepackage{morefloats}
\usepackage{graphicx}
\usepackage{caption}
\usepackage{subcaption}
\usepackage{rotating}
\usepackage[linesnumbered,ruled,vlined]{algorithm2e}
\usepackage[nodots]{numcompress}
\usepackage{booktabs}
\usepackage{epstopdf}
\usepackage{booktabs,caption,fixltx2e}
\usepackage[flushleft]{threeparttable}
\DeclareMathOperator*{\argmin}{\arg\!\min}
\usepackage[toc,page]{appendix}
\usepackage{natbib}
\begin{document}
\nocite{*}

\begin{frontmatter}
\title{Reliability Optimization for series systems under uncertain component reliability in the design phase}
\author[label1]{Qianru Ge}
\author[label1]{Hao Peng}
\author[label1]{Geert-Jan van Houtum}
\author[label1]{Ivo Adan}
\address[label1]{Department of Industrial Engineering and Innovation Sciences, Eindhoven University of Technology, Eindhoven, The Netherlands}

\begin{abstract}
%We consider an OEM who sells a series system for a customer under a Performance-Based Logistic (PBL) contract. During the design phase of the system, the OEM have to select an optimal design for each critical component in the system from all the possible alternatives with uncertain component reliability. The uncertainty  in component reliabilities can lead to large deviations of the realized system availability from the expected system availability. Upon a failure of a critical component in the system, the failed part will be replaced by a as-good-as new component. According to the PBL contract, when the total system down time exceeds a predetermined level, the OEM should pay a penalty cost to the customer with respect to the actual total downtime and a penalty rate. In this case, we formulate the Life Cycle Costs (LCC) of this multi-stage system which are affected by the uncertain component reliability. The LCC consist of design costs, repair costs and downtime costs.
\end{abstract}

\begin{keyword}
Capital goods \sep Reliability optimization  \sep Performance-based contracting \sep Life cycle costs
\end{keyword}
\end{frontmatter}
\section{Introduction}
Capital goods are machines or products that are used by manufacturers to produce their end-products or that are used by service organizations to deliver their services. Advanced technical systems such as medical systems, manufacturing systems, defense systems are examples of capital goods that are critical for the operational processes of their customers. System downtime of these capital goods can cause serious consequences (e.g., millions of euros of reduced production output, extra waiting time of passengers, failure of military missions). Therefore, customers of these complex systems, such as hospitals, militaries and factories, require high availabilities of these systems. On the other hand, the engineering systems involved in capital goods are becoming more and more complex due to the advancement of technologies. The maintenance and repair tasks are too challenging for customers to take care of by themselves. Thus, after-sale services such as maintenances and repairs are often needed by the customers.

%(service contract and performance based contract: Oliva and Kallenberg (2003), Dennis and Kambil (2003), Cohen et al. (2006), Kim et al. (2007, 2011))

As a result, integrating services as a major sustainable source of profit has been widely recommended by a large amount of papers in recent decades for original equipment manufacturers (OEMs). A study conducted by Accenture (\citet{Dennis}) shows that after-sale services contribute only 25\% of the total revenue across all manufacturing companies, but are responsible for $40\%\-50\%$ of the profit. Many OEMs thus have been transforming their business strategies from product-oriented to service-oriented(\citet{Cohen2006}). \citet{Oliva} mentioned that by integrating services into their core product offerings OEMs can get extra revenues, high margins, a higher specialization level, and competitive advantage. According to \citet{Kim2007} and \citet{Oliva}, there are mainly two types of service contract in literature: material based contract and performance based contract. After selling a system, under the traditional material-based contract, the OEM is responsible for the repair of the system only within the warranty period, which is often short compared with the entire life cycle of the system. After the warranty period, the OEM will charge the customer for providing spare parts, maintenance, and other services to have high availability of the system. From customers' perspective, this may lead to higher spare parts costs, repair costs and labor costs, whereas system availabilities can be lower than the anticipation of the customer when buying the system. This undesired situation for the customer can be avoided by making better agreements with the OEM when a new system is being bought.

Performance based contracts (PBC) are novel agreements in the after-sale service market. According to the official U.S. Department of Defense (DoD) guidelines, “the essence of PBC is buying performance outcomes, not the individual parts and repair actions. Instead of buying set levels of spares, repairs, tools, and data, the new focus is on buying a predetermined level of availability to meet the customers' objectives. Therefore under a PBC, customers pay for services according to the system performance. When the system performance fails to meet the predetermined level of availability, OEMs will suffer from a penalty cost.  For instance, since the OEM is responsible to meet a predetermined system availability, if the system fails to meet the system availability target during the service period, e.g., 5-10 years, a penalty cost has to be paid to the customer according to the PBC. \citet{Guajardo} showed that PBC motivates the Rolls-Royce to invest more resources in product reliability and maintenance. \citet{Kim2011} proposed a model in game-theoretic settings to evaluate the trade-off between investing in reliability improvement and in spare assets, so that comparisons can be made regarding the efficiencies of the material-based contract and PBC.  Under the material-based contract, the suppliers invest on the inventory of spares to meet the availability target while under PBC, the suppliers invest on the reliability improvement as well as the spare parts inventory to achieve the same target. The results showed that the material-based contract is less efficient than PBC.

For OEMs,  it is thus important to estimate the system availabilities before the launching of systems. Several methods have been proposed to measure the interval availability during a service period in literature. As defined by \citet{Nakagawa}, interval availability is the fraction of time that a system is operational during a period of time $[0,T]$.

The contribution of the paper is as follows:

Firstly, we consider the uncertainty lies in the failure rates of the components,

\citet{Takacs} first derived the probability distribution function, the limiting distribution and the moments of the interval availability for an on-off stochastic process in close form. Because  it is hard to compute these close-form functions, many numerical methods have  also been developed to calculate the interval availability. \citet{DeSouza} developed two algorithms to compute the cumulative sojourn time of a finite-state Markov chain using the randomization technique. \citet{AlHanbali} developed a numerical approach to evaluate the distribution function of interval availability based on the work of \citet{DeSouza}, and approximated the distribution function by fitting the first two moments of it to a Beta distribution.

Under a PBC, apart from the system availability, the Life cycle cost (LCC) is a major concern of an OEM, which is defined as the total cost  incurred in the design/development, production, operation, maintenance, support, and final disposition of a system over its anticipated useful life span (\citet{Barringer}). In \citet{Oner2007}, a methodology was proposed to measure the LCC of two specific engineer-to-order systems. The results showed that the maintenance cost and downtime cost, which is larger than the acquisition cost, constitute a significant portion of the LCC. These service costs are incurred by system failures, and system failure rates are highly dependent on system designs. Therefore, reliability design decision should be taken under the consideration of LCC and system availability target before investment decisions on new systems are made.
%
%(LCC Life cycle costs:  \"{O}ner et al. (2007)) ({interval availability}: Al Hanbali and van der Heijden (2013); Carrasco (2004); Kirmani and Hood (2008).)

The determination of reliability design aims to get an optimal allocation of component reliabilities while considering the tradeoff between system performance and costs, via enhancing component reliability and building in redundancy (\citet{Kuo}). Reliability design problems belong to the mixed integer nonlinear programming and are known to be NP hard (\citet{Chern}). A lot of works have been done in this area since the 1990s, including multistate system optimization, multi-objective optimization, active and cold-standby redundancy, and so on (\citet{Kuo}). In this paper, we focus on the cases for which redundancy allocation is not an option due to physical constraints and for which component reliabilities can be selected by adopting different materials, designs, or manufacturing processes. For example, \citet{Mettas} estimated the minimum required reliability for each component of a system in order to achieve a system reliability goal with minimum cost. In this paper, the cost function is modeled to take an exponential behavior, which has been introduced by other papers as well, such as \citet{Huang}, \citet{Oner2010} and \citet{Jin}. In \citet{Beraha}, the system reliability was maximized through a random search process that assigns the optimal reliability to each stage of a multi-stage system. \citet{Hwang} used nonlinear programming techniques (sequential unconstrained minimization) to get the optimal reliability allocation in order to maximize the system reliability. \citet{Sheela} solved the reliability allocation problem by sequential weight increasing factor technique to choose the optimal reliability of each component in a system. \citet{Li} developed a 3-level decomposition approach to allocate the resource among subsystems optimally to maximize the system reliability.

Some papers also built the reliability allocation models to find the optimal warranty policies for systems sold with traditional warranty contracts. For example, \citet{Zuo1998} proposed an optimization model to minimize system life cycle cost per unit time subject to reliability requirements, considering warranty cost, maintenance cost, burn-in cost, etc. In \citet{Huang}, a model was developed to compute the optimal warranty period, price and product reliability of a general repairable product sold under a warranty contract, by maximizing the profit under different market situations. \citet{Oner2010} introduced a decision support model to jointly optimize the reliability level and spare parts inventory level of a critical component in the design phase of a system, in order to minimize the system LCC. In this work, repair costs are incurred by both ordinary procedures and emergency procedures, downtime costs are paid to the costumer as a penalty, and design costs are spent for different reliability levels of the components in the design phase. Few papers explicitly model the interval availability in the context of reliability optimization. In \citet{Oner2013}, the authors further considered the redundancy design problem for a multi-component system. A model has been proposed to find the optimal redundancy level, spare parts inventory level as well as spare parts inventory policies. By using this model, the trade-off between the expected uptime and the total cost can be made.  The close form of the expected downtime was given as a measurement of the system interval availability during the contract period. Notice that in this model only the first moment of interval availability has been included.

In most previous works on system reliability optimization problem, all parameters were assumed to be precise. However in reality there is considerable uncertainty and inaccuracy in the estimation of the model parameters, especially the component reliabilities. For newly-developed technologies, the failure mechanisms or the degradation mechanisms may not be completely discovered due to the complexity of a high-tech component. The outcome of a development process for a certain component design can be above the reliability requirement of the component, or below the requirement. This risk from the R\&D process is very high especially for emerging technologies. If we ignore the uncertain component reliabilities, the estimation of the component reliabilities will be wrong. Then we cannot select appropriate components’ designs in an optimal way, in terms of the design costs, penalty costs and other operational costs. For example, if we ignore the risks of obtaining weak components after R\&D processes and just use the reliability estimation (a point estimation) for all components, we may overestimate the reliability of the newly-developed components and that may result in a system design that cannot fulfill the system availability requirement. In this case, deterministic models cannot be applied. To measure the uncertain system reliability due to the insufficient component-level failure data, \citet{Coit1997} used the Kaplan-Meier method to estimate the component reliabilities and their variances to get the confidence interval of the system reliability without considering system cost. This method has been further developed by \citet{Jinandcoit} under the consideration of component failure dependency. Furthermore, \citet{Ramirez-Marquez2008} extended the estimation method of reliability confidence bounds to the multi-state systems, based on uncertain data.

To solve the reliability optimization problem considering component reliability uncertainty, \citet{Murthy1998} developed a model to determine the optimal redundancy design considering quality uncertainty due to manufacture variability, and obtained the best tradeoff between the manufacturing cost and warranty cost of a traditional warranty contract. To capture the uncertainty in the quality of the components, the authors categorized all the components into 3 types: nonconforming item with a very short mean time to failure (MTTF); nonconforming item with a relatively short MTTF; conforming item. The expected warranty cost can be reduced by applying quality control (testing to weed out nonconforming items). Apart from the uncertainty coming from manufacture variability, the uncertainty of component reliability may also stem from the design phase of the component. In \citet{Murthy2003}, the uncertainty in the component reliability has been explicitly characterized by a reliability growth model, in which the uncertain outcome of the R\&D process is modeled by the random failure rate of components. A model has been proposed to find the best tradeoff between the development cost and the resulted reduction on the expected warranty cost of a traditional warranty contract. \citet{Feizollahi} proposed a reliability design framework to address the uncertain component reliability by using the Min-Max regret (also known as robust deviation) approach. The standard nonlinear integer programming formulations were transformed to be linearized binary versions in order to get exact solutions. None of the above works considered the interval availability under a PBC.

\begin{figure}
\centering
\includegraphics[width=1\linewidth]{lr1.pdf}
 \caption{Literature review}
 \label{fig:literature}
\end{figure}

In this paper, we attempt to solve a system reliability design problem considering both uncertain component reliability and interval availability under a PBC, which has not been studied yet. \citet{Jin} proposed a model to minimize the life cycle cost under performance-based contracts in the presence of uncertain system usage.  However, the uncertainty under discussion in our paper is coming from  the design phase of an advanced system. During the design phase of a system, engineers have to select a certain design from all the possible alternatives for each critical component in the system. In real life, the outcome of any development process for a certain design is uncertain with respect to the reliability requirement. For example, since the failure mechanisms of some emerging technologies (e.g., Micro-Electro-Mechanical Systems) are complex, it is often difficult to predict the actual reliability behaviors of the critical components before the development. Therefore newly-designed devices have been found to have different component reliabilities than the expectations after the completion of design. The uncertainty in component reliabilities can lead to large deviations of the realized system availabilities from the expected system availabilities (a point estimate for the system availability). In this case, the uncertainty in component reliabilities also needs to be considered in the decision making of system reliability design under a PBC. To make it more clear about the positioning of our work, we also include Figure \ref{fig:literature} for the readers.



The remainder of the paper is organized as follows:


\section{Notations and assumptions}
 \subsection{Notations}
	 %\begin{center}
   \begin{tabular}{l l}
   %\hline
$i$ & Index of critical components in the system, $i \in \{1,2,...n\}$\\
%$M_{i}$ & Set of rough design alternatives that can be used for component $i$ \\
$j$ & Index of rough design alternatives that can be used for component $i$, \\
    & $j \in \{1,2,...m_{i}\}$.\\
$T$ & The time length of the service contract period\\
$D_0$ & Predetermined accepted total downtime in the service contract\\
$x_{ij}$ & Decision variable of whether choosing the $j$th design of component $i$ or not\\
$c_p$ & Penalty cost per time unit after the total system downtime exceeding $D_0$\\
$c^{ij}_{a}$ & Design cost for the $j$th design of component $i$\\
$c_r^{ij}$ & Repair cost per failure for the $j$th design of component $i$\\
$r_{ij}$ & Repair time per failure for the $j$th design of component $i$\\
$r_{i}(\boldsymbol{x_{i}})$ & Repair time per failure for component $i$\\
$\Lambda_{ij}$ & Failure rate for the $j$th design of component $i$\\
$\Lambda_{i}(\boldsymbol{x_{i}})$  & Failure rate of component $i$\\
%$\Lambda(\boldsymbol{x})$  & System failure rate\\
$\mu_{ij}$ & The expected failure rate for the $j$th design of component $i$ \\
$\mu_{i}(\boldsymbol{x_{i}})$ & The expected failure rate of component $i$ \\
%$\mu(\boldsymbol{x})$ & The expected system failure rate \\
$\sigma_{ij}^{2}$ & The variance of the failure rate for the $j$th design of component $i$\\
$\sigma_{i}^{2}(\boldsymbol{x_{i}})$ &  The variance of the failure rate for component $i$ \\
%$\sigma^{2}(\boldsymbol{x})$ & The variance of the system failure rate \\
%$P(x_{ij},n_{ij})$  & probability of $n_{ij}$ failures occurring in the $j$th design of component $i$ in $[0, T]$ \\
$S(x_{ij})$ & Number of repairs for the $j$th design of component $i$ during $[0,T]$\\
$S_{i}(\boldsymbol{x_{i}})$ & Number of repairs for component $i$ during $[0,T]$\\
%$S(\boldsymbol{x})$ & Total number of failures in the system during $[0,T]$ \\
$A_{i}(\boldsymbol{x_{i}})$  & Design cost of component $i$\\
$A(\boldsymbol{x})$ & Design cost of the system\\
%$N(x_{ij})$ & expected number of repairs occurring in the $j$th design of component $i$ in period $[0, T]$\\
%$R_{ij}(x_{ij})$ &  expected repair cost of the $j$th design of component $i$ in $[0, T]$\\
$R_{i}(\boldsymbol{x_{i}})$ & Expected repair cost of component $i$ in $[0, T]$\\
$R(\boldsymbol{x})$ &  Expected repair cost of the system in $[0, T]$\\
$D_{ij}(x_{ij})$ & Total downtime for the $j$th design of component $i$ in $[0, T]$\\
$D_{i}(\boldsymbol{x_{i}})$& Total downtime of component $i$ in $[0, T]$\\
$D(\boldsymbol{x})$ & Total downtime of the system in $[0, T]$\\
$P(\boldsymbol{x})$ & Expected penalty cost due to downtime exceeding $D_{0}$ in $[0, T]$\\
%$C_f$    & average repair cost in period $[0, T]$

	\end{tabular}
  %\end{center}
	 \subsection{Assumptions}
	
   \begin{enumerate}
   \item The components in discussion are all critical components. If a component fails, the entire system will stop functioning.
  \item The system will be functioning for $T$ years. The exploitation phase is denoted by $[0, T]$. We assume that the system are sold at time $t=0$, and the design costs are also incurred at time $t=0$.
  \item During the exploitation phase of the system $[0, T]$, its total downtime should be less than or equal to $D_0$ years. If the total downtime exceeds $D_0$ years, the OEM will be charged for a penalty cost with respect to the extra downtime and the penalty rate $c_p$.
   \item The failure rates of the components are uncertain in the design phase and distributed with the same general distribution function. Components from different rough designs have different failure rate distribution parameters and design costs. The $j$th design of component $i$ has the failure rate $\Lambda_{ij}$ with mean $\mu_{ij}$ and variance $\sigma_{ij}^2$ and design cost $c_{ij}$.
 \item The life time of each component is independent and exponentially distributed. So the number of the failures for each component over $[0, T]$ has a Poisson distribution with a failure rate $\Lambda_{i}(\boldsymbol{x_{i}})$.
		\item	Once the rough design of a component has been chosen, the value of the component failure rate remains fixed through $[0, T]$.
  \item Once a failure occurs, a repair will be performed. The failed part will be replaced by a ready-to-use part (repair by replacement) from the secondary supplier. The failure rate remains the same after each replacement. The repair cost $c_r^{ij}$ is fixed for each rough design.
 	
	\end{enumerate}
\section{Model description}
During the design phase of a system, engineers have to select a certain design from all the possible alternatives for each critical component in the system. Suppose the system is comprised of $n$ critical components. If one of these critical components fails, the system as a whole stops working. Then the system can be seen to have a series system structure. For each critical component $i \in \{1,2,...,n\}$, one rough design needs to be selected from a set of all the possible alternatives, which is denoted by $\{1,2,...,m_i\}$. Each rough design candidate of component $i$ in the set $\{1,2,...,m_i\}$ has different uncertain reliability parameters and cost parameters. We aim to find out the optimal combination of rough designs for the system to minimize the average total cost over the service period $T$ of a PBL contract.

The lifetimes of the components are assumed to be independent and exponentially distributed. Then for a certain rough design $j$ of component $i$ ($i \in \{1,2,...,n\}$, $j \in \{1,2,...,m_i\}$), we denote its failure rate as $\Lambda_{ij}$, which can fully describe its failure process when its lifetime distribution is exponential. However, as we mentioned in the introduction section, the outcome of any development process for a certain design is uncertain. Therefore, the failure rate $\Lambda_{ij}$ of a certain rough design $j$ for component $i$ is usually not known for sure before the development of the rough design. We use a certain distribution with a probability density function $f_{\Lambda_{ij}}(.)$ or a probability mass function $p_{\Lambda_{ij}}(.)$ to describe the random failure rate $\Lambda_{ij}$ before the development of the rough design, which reflects the prior belief/information about the reliability uncertainty of the technologies used in the rough design. In the evaluation of the average total cost over the service period $T$, these design uncertainties will be taken into account for different combinations of rough designs.

The system will be sold together with a PBL contract over a service period $T$. The OEM is responsible for all the repairs within the service period, as a material-based contract. Moreover, the system availability should be above a predetermined level for the whole service period. Or in other words, the total downtime of the service period should be lower than a predetermined value $D_0$. A penalty cost will be paid by the OEM to compensate the customer if the total downtime exceeds the predetermined level. As a result, the average total cost of a system over $[0, T]$ consists of three parts: (a) design cost, (b) repair cost, and (c) penalty cost. A detailed description of the evaluation of these cost elements are given in the following subsections.



\subsection{Design cost}
	
	Let $c^{ij}_{a}$ denote the cost of designing component $i$ according to a rough design $j$ ($i \in \{1,2,...,n\}$, $j \in \{1,2,...,m_i\}$). It includes all the costs incurred to realize a certain rough design of component during the design phase, e.g., human resources, experimental equipment, testing or prototype units, etc. Define binary decision variable $x_{ij}$ as
	\[ x_{ij} = \left \{
	  \begin{array} {l l}
		0 & \quad \text {not selecting rough design $j$ for component $i$, } \\
		1 & \quad \text {selecting rough design $j$ for component $i$. }
		\end {array} \right.\]
		Then the design cost for component $i$ is given by
	 \begin{eqnarray}
	A_{i}(\boldsymbol{x_{i}})=\sum^{m_{i}}_{j=1} {c^{ij}_{a} x_{ij}},
\label{Ai}
		 \end{eqnarray}
		 where $\boldsymbol{x_{i}}=[x_{i1},x_{i2},...,x_{i,m_i}]$ represents the selection of rough designs for component $i$. Since we assume the OEM can only select one rough design from all the possible candidates for each critical component, $\sum^{m_{i}}_{j=1}{x_{ij}=1}$.
And the total design cost for the system is given by
\begin{eqnarray}
	A(\boldsymbol{x})=\sum ^{n}_{i=1}A_{i}(\boldsymbol{x_{i}})=\sum ^{n}_{i=1}\sum^{m_{i}}_{j=1} {c^{ij}_{a} x_{ij}}, \label{A}
\end{eqnarray}
where $\boldsymbol{x}$ represents the selection plan of rough designs for all the critical components in the system.
	
\subsection{Repair cost}
	
When a failure occurs in period $[0, T]$, a repair will be performed by the OEM. We assign $c_r^{ij}$ as the repair cost for each failure of the $j$th rough design for component $i$ ($i \in \{1,2,...,n\}$, $j \in \{1,2,...,m_i\}$) respectively. The repair cost $c_r^{ij}$ corresponds to diagnosis cost, replacement cost, and other service costs for each repair. A failure-based policy for maintenance is assumed for this multi-component system in order to evaluate the maintenance cost over the service period $T$. Some other preventive maintenance policies, such as age/time-based policies or condition-based policies, can also be applied to the system, which may result in lower maintenance costs. However, at the design phase, it is usually hard to make a decision on the maintenance policies that are at the operational level. The evaluation of repair cost based on a failure-based policy is relatively accurate and conservative. Let $S_{i}(\boldsymbol{x_{i}})$ denote the total number of repairs for component $i$ during $[0,T]$. Under such a failure-based policy and the assumption that the lifetimes of components are exponentially distributed, the expected number of repairs for component $i$ during $[0,T]$, $\mathbb{E}[S_{i}(\boldsymbol{x_{i}})]$, equals the product of the failure rate and the service period $\Lambda_i(\boldsymbol{x_{i}}) T$ (Barlow and Proschan 1967), which is still a random variable in our formulation due to the randomness of failure rate $\Lambda_{i}(\boldsymbol{x_{i}})$. The the expectation of $\mathbb{E}[S_{i}(\boldsymbol{x_{i}})]$ is given by
%$E[N_{i}(T)]$
\begin {eqnarray}
\mathbb{E}_{\Lambda_{i}(\boldsymbol{x_{i}})} \bigg\{ \mathbb{E}\bigg[S_{i}(\boldsymbol{x_{i}})\bigg] \bigg\}=\mathbb{E}\bigg[\Lambda_{i}(\boldsymbol{x_{i}})T\bigg] =\mathbb{E}\bigg(\sum_{j=1}^{m_{i}}{\Lambda_{ij}x_{ij}}T\bigg)=\sum_{j=1}^{m_{i}}{\mu_{ij}x_{ij}}T,
\end {eqnarray}
where $\mathbb{E}_{\Lambda_{i}(\boldsymbol{x_{i}})}$ denotes the expectation over the distribution of $\Lambda_{i}(\boldsymbol{x_{i}})$, $\Lambda_{i}(\boldsymbol{x_{i}})=\sum_{j=1}^{m_{i}}{\Lambda_{ij}x_{ij}}$ is the random failure rate of component $i$ given a certain rough design $\boldsymbol{x_i}$, and $\mu_{ij}$ is the mean value of $\Lambda_{ij}$. Then the expected repair cost for component $i$, $R_{i}(\boldsymbol{x_{i}})$, is the product of the repair cost per failure and the expected number of repairs, which is also a random variable. The expectation of $R_{i}(\boldsymbol{x_{i}})$ over the distribution of $\Lambda_i(\boldsymbol{x_{i}})$ is given as
\begin {eqnarray}
\mathbb{E}_{\Lambda_{i}(\boldsymbol{x_{i}})} \bigg[ R_{i}(\boldsymbol{x_{i}}) \bigg]=\mathbb{E}\bigg(\sum_{j=1}^{m_{i}}{\Lambda_{ij}c_r^{ij}x_{ij}}T\bigg)= \sum_{j=1}^{m_{i}}{\mu_{ij}c_r^{ij}x_{ij}}T. \label{ERi}
\end {eqnarray}

Given that the failure processes of all the critical components are independent of each other, the expected system repair cost $R(\boldsymbol{x})$ in $[0,T]$ is the sum of $R_{i}(\boldsymbol{x_i}), \forall i \in \{1,2,...,n\}$. The expectation of $R(\boldsymbol{x})$ over all the distributions of $\Lambda_{1}(\boldsymbol{x_{1}}),\Lambda_{2}(\boldsymbol{x_{2}}),..., \Lambda_{n}(\boldsymbol{x_{n}})$ is given as
\begin {eqnarray}
\mathbb{E}_{\Lambda(\boldsymbol{x})} \bigg[ R(\boldsymbol{x}) \bigg]= \mathbb{E}_{\Lambda(\boldsymbol{x})} \bigg[ \sum_{i=1}^{n}R_{i}(\boldsymbol{x_{i}}) \bigg]=\sum_{i=1}^{n}\sum_{j=1}^{m_{i}}{\mu_{ij}c_r^{ij}x_{ij}}T. \label{ER}
\end {eqnarray}
where $\mathbb{E}_{\Lambda(\boldsymbol{x})}$ denotes the expectation over all the independent distributions of $\Lambda_{1}(\boldsymbol{x_{1}}),\Lambda_{2}(\boldsymbol{x_{2}}),..., \Lambda_{n}(\boldsymbol{x_{n}})$.
The variance of the expected system repair cost with respect to the random failure rates can be expressed as
\begin {eqnarray}
 Var_{\Lambda(\boldsymbol{x})} \bigg[ R(\boldsymbol{x}) \bigg]  &=&Var_{\Lambda(\boldsymbol{x})} \bigg[ \sum_{i=1}^{n}R_{i}(\boldsymbol{x_{i}}) \bigg]
= \sum_{i=1}^{n} Var_{\Lambda_i(\boldsymbol{x_{i}})}\bigg[R_{i}(\boldsymbol{x_i})\bigg] \nonumber \\
 &=& \sum_{i=1}^{n} Var_{\Lambda_i(\boldsymbol{x_{i}})}(\sum_{j=1}^{m_{i}}{\Lambda_{ij}c_r^{ij}x_{ij}}T) \nonumber\\
&=& \sum_{i=1}^{n} \sum_{j=1}^{m_{i}}{(c_r^{ij}x_{ij}}T)^2 Var(\Lambda_{ij}).
\end {eqnarray}

\subsection{Penalty cost}
A period of system downtime $r_{ij}(r_{ij}<<T)$ will be incurred due to a random failure of component $i$ with rough design $j$ in the system ($i \in \{1,2,...,n\}$, $j \in \{1,2,...,m_i\}$). Notice that while evaluating the repair costs in the previous section we ignore the downtime since the downtime is usually negligible compared with the service period $T$. However, under a PBL contract, when the total system downtime over the service period $T$ exceeds a predetermined value $D_0$, a penalty cost should be paid by the OEM to customers with a rate $c_p$. The predetermined target value $D_0$ is at the same scale as $r_{ij}$. We assume the system downtime of each failure varies among different components with different rough designs. Hence the total system downtime $D{(\boldsymbol{x})}$ over the service period $T$ is dependent on the number of failures $S_{i}(\boldsymbol{x_{i}})$ and the repair time per failure $r_{i}(\boldsymbol{x_{i}})$ for component $i$,$\forall i\in \{1,\dots,n\}$ in $[0,T]$. The repair time per failure for component $i$, $r_{i}(\boldsymbol{x_{i}})=\sum_{j=1}^{m_{i}}{r_{ij}x_{ij}}$, is a fixed value after the selection plan for component $i$ has been made. Notice that the number of failures from component $i$, $S_{i}(\boldsymbol{x_i})$, is a Poisson distributed random variable, whose distribution is given as
%\begin {eqnarray}
%Pr\bigg[S_{i}(\boldsymbol{x_i})=s_{i}\bigg]&=&\frac{e^{-\Lambda_{i}(\boldsymbol x_{i})T}}{s_{i}!}{\bigg[\Lambda_{i}(\boldsymbol{x_{i}})T\bigg]}^{s_{i}} \nonumber\\
%&=& \frac{e^{-\sum_{j=1}^{m_{i}}{\Lambda_{ij}x_{ij}T}}(\sum_{j=1}^{m_{i}}{\Lambda_{ij}x_{ij}T})^{s_{i}}}{s_{i}!}.
%\end {eqnarray}
\begin{eqnarray}
Pr\bigg[S_{i}(\boldsymbol{x_i})=s_{i}\bigg] &=&\sum^{m_{i}}_{j=1}{x_{ij}Pr(S_{ij}=s_{i})} \nonumber\\
&=&\sum^{m_{i}}_{j=1}{x_{ij}\int^{\infty}_{0}Pr(S_{ij}=s_{i}|\Lambda_{ij}=\lambda_{ij})f_{\Lambda_{ij}}(\lambda_{ij})d\lambda_{ij}} \nonumber\\
&=&\sum^{m_{i}}_{j=1}{x_{ij}\int^{\infty}_{0}{\frac{e^{-\lambda_{ij}T}(\lambda_{ij}T)^{s_{i}}}{s_{i}!}f_{\Lambda_{ij}}(\lambda_{ij})}d\lambda_{ij}}
\end{eqnarray}
And the system downtime can be expressed as
\begin{eqnarray}
D(\boldsymbol{x})= r_{1}(\boldsymbol{x_{1}})S_{1}(\boldsymbol{x_{1}})+r_{2}(\boldsymbol{x_{2}})S_{2}(\boldsymbol{x_{2}})+\ldots+r_{n}(\boldsymbol{x_{n}})S_{n}(\boldsymbol{x_{n}}).
\label{D}
\end{eqnarray}

According to the service contract, the OEM should pay a penalty cost to the customer. Since the failure rates $\Lambda_{1}(\boldsymbol{x_{1}}),\Lambda_{2}(\boldsymbol{x_{2}}),...,\Lambda_{n}(\boldsymbol{x_{n}})$ are random variables, the expected penalty cost is a random variable as well. Without loss of generality, we assume the failure rates are continuous random variables, with probability density functions $f_{\Lambda_{ij}}(.)$ over region $\mathcal{O}_{ij}$. Then the expected penalty cost due to extra downtime exceeding $D_{0}$ is given as
%Then for component $i$ with a certain design $\boldsymbol{x_{i}}$, the probability density function of the failure rate is $f_{\Lambda_{i}(\boldsymbol{x_{i}})}(.)$ over region $\mathcal{O}_{i}(\boldsymbol{x_{i}})$ The expectation of $P(\boldsymbol{x})$ can be expressed as when $D(\boldsymbol{x})$ exceeds the predetermined target downtime $D_{0}$.
\begin{eqnarray}
P(\boldsymbol{x})& = & \mathbb{E}_{S}\bigg\{\bigg[D(\boldsymbol{x})-D_{0}\bigg]^{+} c_{p} \bigg\} \nonumber\\
&=& \mathbb{E}_{S}\bigg\{\bigg[\sum_{i=1}^{n}{S_{i}(\boldsymbol{x_{i}})r_{i}(\boldsymbol{x_i})}-D_{0}\bigg]^{+} c_{p} \bigg\} \nonumber\\
&=&\sum_{s_{1}=0}^{\infty}{Pr\bigg[S_{1}(\boldsymbol{x_{1}})=s_{1}\bigg]}\dots\sum_{s_{n}=0}^{\infty}{Pr\bigg[S_{n}(\boldsymbol{x_{n}})=s_{n}\bigg]} \bigg[\sum_{i=1}^{n}{s_{i}r_{i}(\boldsymbol{x_{i}})}-D_{0}\bigg]^{+}c_{p} \nonumber\\
&=&\sum^{\infty}_{s_{1}=0}\sum^{m_{1}}_{j=1}{x_{1j}}\int^{\infty}_{0}{Pr(S_{1j}=s_{1}|\Lambda_{1j}=\lambda_{1j})f_{\Lambda_{1j}}(\lambda_{1j})d\lambda_{1j}}\dots \nonumber\\
&&\sum^{\infty}_{s_{n}=0}\sum^{m_{n}}_{j=1}{x_{nj}}\int^{\infty}_{0}{Pr(S_{nj}=s_{n}|\Lambda_{ij}=\lambda_{ij})f_{\Lambda_{ij}}(\lambda_{ij})d\lambda_{ij}}\bigg[\sum_{i=1}^{n}{s_{i}r_{i}(\boldsymbol{x_{i}})}-D_{0}\bigg]^{+}c_{p} \nonumber\\
&=& \sum^{\infty}_{s_{1}=0}\sum^{m_{1}}_{j=1}{x_{1j}}\int^{\infty}_{0}{\frac{e^{-\lambda_{1j}T}(\lambda_{1j}T)^{s_{1}}}{s_{1}!}f_{\Lambda_{1j}}(\lambda_{1j})d\lambda_{1j}}\dots \nonumber\\
&&\sum^{\infty}_{s_{n}=0}\sum^{m_{n}}_{j=1}{x_{nj}}\int^{\infty}_{0}{\frac{e^{-\lambda_{nj}T}(\lambda_{nj}T)^{s_{n}}}{s_{n}!}f_{\Lambda_{nj}}(\lambda_{nj})d\lambda_{nj}}\bigg[\sum_{i=1}^{n}{s_{i}r_{i}(\boldsymbol{x_{i}})}-D_{0}\bigg]^{+}c_{p} \nonumber\\
&=&\int^{\infty}_{0}\dots\int^{\infty}_{0}{\sum_{s_{1}=0}^{\infty}\sum^{m_{1}}_{j=1}{x_{1j}\frac{e^{\lambda_{1}T}(\lambda_{1}T)^{s_{1}}}{s_{1}!}\dots \sum_{s_{n}=0}^{\infty}\sum^{m_{n}}_{j=1}{x_{nj}\frac{e^{\lambda_{n}T}(\lambda_{n}T)^{s_{n}}}{s_{n}!}}}}\nonumber\\
&&\bigg[\sum_{i=1}^{n}{s_{i}r_{i}(\boldsymbol{x_{i}})}-D_{0}\bigg]^{+}c_{p}d\lambda_{1j}\dots d\lambda_{nj}
\end{eqnarray}
%\sum_{s_{1}=0}^{\infty}{\frac{e^{-\Lambda_{1}(\boldsymbol{x_{1}})T}(\Lambda_{1}(\boldsymbol{x_{1}})T)^{s_{1}}}{s_{1}!}}\dots\sum_{s_{n}=0}^{\infty}{\frac{e^{-\Lambda_{n}(\boldsymbol{x_{n}})T}(\Lambda_{n}(\boldsymbol{x_{n}})T)^{s_{n}}}{s_{n}!}} \nonumber\\
% &&\bigg[\sum_{i=1}^{n}{s_{i}r_{i}(\boldsymbol{x_{i}})}-D_{0}\bigg]^{+}c_{p}.
%\end{eqnarray}


%\begin{eqnarray}
%&& \mathbb{E}_{\Lambda(\boldsymbol{x})} \bigg[P(\boldsymbol{x}) \bigg]=
%\idotsint_{\lambda_{i} \in \mathcal{O}_{i}(\boldsymbol{x_i})} \sum_{s_{1}=0}^{\infty}{\frac{e^{-\lambda_{1}T}(\lambda_{1}T)^{s_{1}}}{s_{1}!}}\dots\sum_{s_{n}=0}^{\infty}{\frac{e^{-\lambda_{n}T}(\lambda_{n}T)^{s_{n}}}{s_{n}!}} \nonumber\\
%&&\bigg[\sum_{i=1}^{n}{s_{i}r_{i}(\boldsymbol{x_{i}})}-D_{0}\bigg]^{+}c_{p}\prod_{i=1}^{n}{f_{\Lambda_{i}(\boldsymbol{x_{i}})}(\lambda_{i})d\lambda_{1}\dots d\lambda_{n}}. \label{EP2}
%\end{eqnarray}
%The variance of $P(\boldsymbol{x})$ can be expressed as
%\begin{eqnarray}
%&&Var_{\Lambda(\boldsymbol{x})} \bigg[ P(\boldsymbol{x}) \bigg]=
%\idotsint_{\lambda_{i} \in \mathcal{O}_{i}(\boldsymbol{x_{i}})} \sum_{s_{1}=0}^{\infty}{\frac{e^{-\lambda_{1}T}(\lambda_{1}T)^{s_{1}}}{s_{1}!}}\dots\sum_{s_{n}=0}^{\infty}{\frac{e^{-\lambda_{n}T}(\lambda_{n}T)^{s_{n}}}{s_{n}!}}\nonumber\\
%&&\bigg\{\bigg[\sum_{i=1}^{n}{s_{i}r_{i}(\boldsymbol{x_{i}})}-D_{0}\bigg]^{+}c_{p} -\mathbb{E}_{\Lambda(\boldsymbol{x})} \bigg[ P(\boldsymbol{x}) \bigg]\bigg\}^{2} \prod_{i=1}^{n} f_{\Lambda_{i}(\boldsymbol{x_{i}})}(\lambda_{i}) \ d\lambda_{1}\dots d\lambda_{n}.
%\end{eqnarray}

\subsection{Optimization model}
The OEM is interested in minimizing the expected total life cycle cost $\pi(\boldsymbol{x})$, which is the sum of the total design cost $A(\boldsymbol{x})$, the expected system repair cost $R(\boldsymbol{x})$ and the expected penalty cost $P(\boldsymbol{x})$. Due to the randomness of the failure rates $\Lambda_{1}(\boldsymbol{x_{1}}),\Lambda_{2}(\boldsymbol{x_{2}}),...,\Lambda_{n}(\boldsymbol{x_{n}})$ in rough designs, the expected total life cycle cost $\pi(\boldsymbol{x})$ is random. If the decision maker is risk-neutral, the optimization model of this problem can be formulated as
\begin{eqnarray}
\text{(P)} \hspace{15mm} & min_{\boldsymbol{x}} & \hspace{10mm} \mathbb{E}_{\Lambda(\boldsymbol{x})} \bigg[ \pi(\boldsymbol{x})\bigg] \nonumber\\
& \text{ s.t. }&  \hspace{5mm} \sum_{j=1}^{m_i}{x_{ij}}=1, \hspace{10mm} \text{for all} i \in \{1,2,...,n\} \nonumber\\
& & \hspace{5mm} x_{ij} \in \{0,1\}, \hspace{10mm}  \text{for all} i \in \{1,2,...,n\}, j \in \{1,2,...,m_i\}
\end{eqnarray}
where $\pi(\boldsymbol{x})=A(\boldsymbol{x})+R(\boldsymbol{x})+P(\boldsymbol{x})$. Notice that in the above optimization formulation, although we do not include a probability constraint on the distribution of $\pi(\boldsymbol{x})$, the variance of $\pi(\boldsymbol{x})$ still gets penalized by the third cost term $P(\boldsymbol{x})$ in the objective function. This optimization problem is difficult to solve because of the complicated form of the objective function. In the objective function, the expected penalty cost $P(\boldsymbol{x})$ is a multiple integration over the ranges of the random failure rates, which is difficult to calculate. Therefore, in the next section, an approximation method will be proposed to make the evaluation of the objective function easier.

%\paragraph{\textbf{Proposition 1}} \textit{For component $i$, if there exist indices $k$ and $l\in J_{i}$ such that $\mu_{ik}\leq \mu_{il}$, $\sigma_{ik}\leq \sigma_{il}$, $c^{ik}_{a}\geq c^{il}_{a} $, $c^{ik}_{r}\geq c^{il}_{r} $ and $r_{ik} \leq r_{il}$ },  then the optimal solution of the problem (P) has $x_{il}=0$
%
%\paragraph{\textbf{Proof}} The proof is given in \ref{Pro1}.


\section{Approximate evaluation}

In this section, we will describe how to evaluate a given selection plan of rough designs for all the critical components in the system approximately. For a given policy $\boldsymbol{x}=[\boldsymbol{x_{1}},\boldsymbol{x_{2}},\dots,\boldsymbol{x_{n}}]$, the design cost $A(\boldsymbol{x})$ and the expected repair cost $\mathbb{E}_{\Lambda(\boldsymbol{x})} \big[ R(\boldsymbol{x}) \big]$ can be determined from Equation \eqref{A} and \eqref{ER}. These two cost terms are linear functions of the decision variables $\boldsymbol{x}=[\boldsymbol{x_{1}},\boldsymbol{x_{2}},\dots,\boldsymbol{x_{n}}]$.


For the exact evaluation of the expected penalty cost $\mathbb{E}_{\Lambda(\boldsymbol{x})}\big[P(\boldsymbol{x})\big]$ of a given policy $\boldsymbol{x}$, we will suffer from the ``curse of dimensionality" when the number of critical components becomes large, since each critical component contributes a dimension in computing the convolution and integration in Equation \eqref{EP2}. The computation time will become intractable as the number of cirtical components grows. (In a special case, if the repair time $r$ for each rough design of each component is the same, the system downtime equals the product of the number of failures of the system and $r$. In this case, we only need to compute the integration part. But we will still face the ``curse of dimensionality" in computing the integration part.) With the consideration of the real life situation, we introduce an approximation method to help estimate the expected penalty cost. Remember that the expected penalty cost is the expected exceeding downtime multiplied with the penalty cost rate, i.e., $\big[D(\boldsymbol{x})-D_0\big]^{+}c_p$. Then if we can find an approximation method to calculate the distribution of the system downtime $D(\boldsymbol{x})$ efficiently, it will become relatively easy to evaluate the expected penalty cost in the objective function.


\subsection{Two-moment fits of the downtime distributions}

As we mentioned before, the total downtime $D(\boldsymbol{x})$ is the summation of the downtimes for all critical components and the number of failures for each component is Poisson distributed. In the approximate evaluation, we approximate the downtime by fitting a mixed Erlang distribution to the first two moments of $D(\boldsymbol{x})$. From Equation \eqref{D}, we can obtain the first moment, variance and second moment of $D(\boldsymbol{x})$ over the distributions of $S_{i}(\boldsymbol{x_{i}})$ ($i=1,2,\dots,n$) are given by
\begin{eqnarray}
\mathbb{E}_{S}\bigg[D(\boldsymbol{x})\bigg]=\mathbb{E}_{S}\bigg[\sum_{i=1}^{n}{r_{i}(\boldsymbol{x_{i}})S_{i}(\boldsymbol{x_{i}})}\bigg]=\sum_{i=1}^{n}{r_{i}(\boldsymbol{x_{i}})\Lambda_{i}(\boldsymbol{x_{i}})T},
\label{ED1}\\
Var_{S}\bigg[D(\boldsymbol{x})\bigg]=\sum_{i=1}^{n}{Var_{S}\bigg[{r_{i}(\boldsymbol{x_{i}})S_{i}(\boldsymbol{x_{i}})}\bigg]}=\sum_{i=1}^{n}{r_{i}^{2}(\boldsymbol{x_{i}})\Lambda_{i}(\boldsymbol{x_{i}})T},
\label{VD1}\\
\mathbb{E}_{S}\bigg[D^{2}(\boldsymbol{x}) \bigg]= \bigg[\sum_{i=1}^{n}{r_{i}(\boldsymbol{x_{i}})\Lambda_{i}(\boldsymbol{x_{i}})T}\bigg]^{2} + \sum_{i=1}^{n}{r_{i}^{2}(\boldsymbol{x_{i}})\Lambda_{i}(\boldsymbol{x_{i}})T}.
\end{eqnarray}

The first moment $\mu_{D}$, variance $\sigma^{2}_{D}$ and the coefficient of variation $c_{v}$  of $D(\boldsymbol{x})$ over the distributions of $S_{i}(\boldsymbol{x_{i}})$ and $\Lambda_{i}(\boldsymbol{x_{i}})$ are given by
\begin{eqnarray}
\mu_{D}(\boldsymbol{x})&=&\mathbb{E}_{\Lambda(\boldsymbol{x})}\bigg[\sum_{i=1}^{n}{r_{i}(\boldsymbol{x_{i}})\Lambda_{i}(\boldsymbol{x_{i}})T}\bigg]=T\sum_{i=1}^{n}{r_{i}(\boldsymbol{x_i})\mu_{i}(\boldsymbol{x_i})},
\label{muD}\\
\sigma^{2}_{D}(\boldsymbol{x})&=&\mathbb{E}_{\Lambda(\boldsymbol{x})}\mathbb{E}_{S} \bigg[D^{2}(\boldsymbol{x}) \bigg]- \bigg\{\mathbb{E}_{\Lambda(\boldsymbol{x})}\mathbb{E}_{S}\bigg[D(\boldsymbol{x}) \bigg] \bigg\}^{2} \nonumber\\
&=& \mathbb{E}_{\Lambda(\boldsymbol{x})}\bigg[\bigg[\sum_{i=1}^{n}{r_{i}(\boldsymbol{x_{i}})\Lambda_{i}(\boldsymbol{x_{i}})T}\bigg]^{2} +\sum_{i=1}^{n}{r_{i}^{2}(\boldsymbol{x_{i}})\Lambda_{i}(\boldsymbol{x_{i}})T}\bigg]-\bigg[T\sum_{i=1}^{n}{r_{i}(\boldsymbol{x_i})\mu_{i}(\boldsymbol{x_i})}\bigg]^{2}\nonumber\\
&=&T^{2}\sum_{i=1}^{n}{r_{i}^{2}(\boldsymbol{x_i})\sigma_{i}^{2}(\boldsymbol{x_i})}+T\sum_{i=1}^{n}{r_{i}^{2}(\boldsymbol{x_i})\mu_{i}(\boldsymbol{x_i})}
\label{sigma}\\
c_{v}(\boldsymbol{x}) &=& \frac{\sigma_{D}(\boldsymbol{x})}{\mu_{D}(\boldsymbol{x})}
\end{eqnarray}

where $\mu_{i}(\boldsymbol{x_i})$ and $\sigma_{i}(\boldsymbol{x_i})$ is the mean and standard deviation of $\Lambda_{i}(\boldsymbol{x_i})$ respectively.
%%\mu_{D^{2}}(\boldsymbol{x})&=&
%T^{2}\sum_{i=1}^{n}r_{i}(\boldsymbol{x_i})^{2}\int_{\lambda_{i}(\boldsymbol{x_i})\in\Lambda_{i}(\boldsymbol{x_i})}{\lambda_{i}(\boldsymbol{x_i})^{2}f(\lambda_{i}(\boldsymbol{x_i}))d\lambda_{i}(\boldsymbol{x_i})}\nonumber\\
%&&+2T^{2}\sum_{i=1}^{n}\sum_{i\ne j}{r_{i}(\boldsymbol{x_i})r_{j}(\boldsymbol{x_j})\mu_{i}(\boldsymbol{x_i})\mu_{j}(\boldsymbol{x_j})}\nonumber\\
%&&+T\sum_{i=1}^{n}{r_{i}(\boldsymbol{x_i})^{2}\mu_{i}(\boldsymbol{x_i})}


Given that $\mu_{D}>0$, and $0<c_{v}\leq 1$, according to Tijms 1994, we fit the downtime distribution to an Erlang$(k-1,k)$ distribution with parameters$(k,\theta,q_{E})$ such that the first two moments of $D(\boldsymbol{x})$ equal the first two moments of the Erlang $(k-1,k)$ distribution. Thus the parameters of the Erlang $(k-1,k)$ distribution can be obtained as
\begin{eqnarray}
&&k(\boldsymbol{x}) = \lceil \frac{1}{c_{v}^{2}(\boldsymbol{x})} \rceil, \label{k1}\\
&&q_{E}(\boldsymbol{x})= \frac{1}{1+c^{2}_{v}(\boldsymbol{x})}\bigg[k(\boldsymbol{x})c^{2}_{v}(\boldsymbol{x})-\sqrt{k(\boldsymbol{x})\big[1+c^{2}_{v}(\boldsymbol{x})\big]-k^{2}(\boldsymbol{x})c^{2}_{v}(\boldsymbol{x})} \bigg], \label{q1}\\
&&\theta(\boldsymbol{x}) = \frac{k(\boldsymbol{x})-q_{E}(\boldsymbol{x})}{\mu_{D}(\boldsymbol{x})}. \label{theta1}
\end{eqnarray}

 Then for random failure rates $\Lambda_{1}(\boldsymbol{x_{1}})$, $\Lambda_{2}(\boldsymbol{x_{2}})$,..., $\Lambda_{n}(\boldsymbol{x_{n}})$ the expected exceeding total downtime can be approximated by
\begin{eqnarray}
&&D_{E_{1}}^{A}(\boldsymbol{x})=\mathbb{E}_{\Lambda}\mathbb{E}_{S}\bigg\{\big[D(\boldsymbol{x})-D_{0}\big]^{+}\bigg\} \nonumber\\
%&&=(\frac{k-q}{\theta}-D_{0})\sum_{j=0}^{k-1}{\frac{(\theta(\boldsymbol{x})D_{0})^j}{j!}e^{-\theta(\boldsymbol{x})D_{0}}}\nonumber\\
&&=\bigg[\frac{k(\boldsymbol{x})-q_{E}(\boldsymbol{x})}{\theta(\boldsymbol{x})}-D_{0}\bigg]\sum_{j=0}^{k(\boldsymbol{x})-2}{\frac{\big[\theta(\boldsymbol{x}) D_{0}\big]^j}{j!}e^{-\theta(\boldsymbol{x})D_{0}}}\nonumber\\
&&+\bigg[\frac{k(\boldsymbol{x})-q_{E}(\boldsymbol{x})}{\theta(\boldsymbol{x})}\bigg]\frac{\big[\theta(\boldsymbol{x})D_{0}\big]^{k(\boldsymbol{x})-1}}{\big[k(\boldsymbol{x})-1\big]!}e^{-\theta(\boldsymbol{x})D_{0}}.
\label{EXD1}
\end{eqnarray}
The derivation is given in Appendix \ref{AG}.


If $c_{v} \geq 1$, we fit the downtime distribution to an Hyperexponential distribution with parameters$(\theta_{1},\theta_{2}, q_{H})$ such that the first two moments of $D(\boldsymbol{x})$ equal the first two moments of the Hyperexponential distribution. Thus the parameters of the Hyperexponential distribution can be obtained as
\begin{eqnarray}
     \theta_{1}(\boldsymbol{x}) &=& \frac{2}{\mu_{D}(\boldsymbol{x})}\bigg(1+\sqrt{\frac{c^{2}_{v}(\boldsymbol{x})-\frac{1}{2}}{c^{2}_{v}(\boldsymbol{x})+1}} \bigg), \label{theta11}\\
     \theta_{2}(\boldsymbol{x}) &=& \frac{4}{\mu_{D}(\boldsymbol{x})} - \theta_{1}(\boldsymbol{x}), \label{theta12}\\
     q_{H}(\boldsymbol{x}) &=& \frac{\theta_1(\boldsymbol{x})(\theta_2(\boldsymbol{x})\mu_{D}(\boldsymbol{x})-1)}{\theta_2(\boldsymbol{x}) -\theta_1(\boldsymbol{x})} \label{q2}
\end{eqnarray}
Then the expected exceeding total downtime can be approximated by
\begin{eqnarray}
D_{E_{2}}^{A}(\boldsymbol{x}) =\frac{q_{H}(\boldsymbol{x})}{\theta_{1}(\boldsymbol{x})} e^{-\theta_{1}(\boldsymbol{x}) D_0 } + \frac{1-q_{H}(\boldsymbol{x})}{\theta_{2}(\boldsymbol{x})} e^{-\theta_{2}(\boldsymbol{x}) D_0} \label{EXD2}
\end{eqnarray}
The derivation can be found in Appendix \ref{AG}.


The procedure of the approximation evaluation method is illustrated in the following algorithm.

\textbf{Algorithm 1}
\label{Algorithm1}
\begin{description}
\item[Step 1] Compute the $\mu_{D}(\boldsymbol{x})$, $\sigma^{2}_{D}(\boldsymbol{x})$ and $c_{v}(\boldsymbol{x})$ of the downtime distribution, check the value of $c_{v}(\boldsymbol{x})$: if $0<c_{v}(\boldsymbol{x}) \leq 1$, go to step 2; if $c_{v}(\boldsymbol{x}) > 1$, go to step 3.
\item[Step 2] Fit the first two moments of an Erlang$(k-1, k)$ distribution to be equal to the first two moments of the downtime distribution.
 \begin{description}
 \item[Step 2-a] Let $k(\boldsymbol{x})$, $\theta(\boldsymbol{x})$ and $q_{E}(\boldsymbol{x})$ be the parameters of the fitted Erlang$(k-1,k)$ distribution, and compute the values of $k(\boldsymbol{x})$, $\theta(\boldsymbol{x})$ and $q_{E}(\boldsymbol{x})$ according to Equation \eqref{k1},\eqref{q1} and \eqref{theta1}.
 \item[Step 2-b] Calculate the expected exceeding total downtime $D_{E1}^A(\boldsymbol{x})$ according to Equation \eqref{EXD1}.
\end{description}
\item[Step 3] Fit the first two moments of a Hyperexponential distribution to be equal to the first two moments of the downtime distribution.

\begin{description}
 \item[Step 3-a] Let $\theta_{1}(\boldsymbol{x})$, $\theta_{2}(\boldsymbol{x})$ and $q_{H}(\boldsymbol{x})$ be the parameters of the fitted Hyperexponential distribution, compute the values of these parameters according to Equation \eqref{theta11}, \eqref{theta12} and \eqref{q2}.
 \item[Step 3-b] Compute the expected exceeding total downtime $D_{E2}^A(\boldsymbol{x})$ according to Equation \eqref{EXD2} .
\end{description}

\end{description}

\subsection{Accuracy of the approximation}
To assess the accuracy of the approximation of the expected exceeding downtime $D_{E}^{A}(\boldsymbol{x})$, we use Monte Carlo simulation method to generate the simulation results $D_{E}^{S}(\boldsymbol{x})$ according to \ref{MCP}, so that the comparison between $D_{E}^{A}(\boldsymbol{x})$ and $D_{E}^{S}(\boldsymbol{x})$ can be made. A full factorial test bed is set up to show the accuracy of the approximation under different parameter settings. We investigate the effect of three factors: the number of the critical components $n$, standard deviation $\sigma_{i}(\boldsymbol{x_i})$ of the failure rate for each component and the predetermined targeted downtime $D_{0}$. Notice that although the standard deviation $\sigma_i$ is a function of the design $\boldsymbol{x_i}$, we will treat it as a parameter of the approximation model or simulation model to assess the approximation accuracy without loss of generality. Similarly, for $\mu_i(\boldsymbol{x_i})$ and $r_i(\boldsymbol{x_i})$ in the calculation of $D_{E}^{A}(\boldsymbol{x})$ and $D_{E}^{S}(\boldsymbol{x})$, we will also treat them as parameters ($\mu_i , r_i$) of the approximation model or simulation model.

The instance space $\Omega$ is thus defined by all combinations of the choices for the three factors, i.e., $(n_{\alpha},\sigma_{\beta},D_{\gamma}) \in \Omega$, $\forall \alpha \in \{5,50,100\}, \beta \in \{0.2,0.35,0.5\}, \gamma \in \{01,02,03,04\}$. The values of choices for each factor are shown in Table \ref{tab:testbedps}. In the test bed, the factor $D_{\gamma}$ is a coefficient to generate different values of $D_0$ by the following expression,
 \begin{eqnarray}
 D_{0} = D_{\gamma}\sum_{i=1}^{n}{\mu_{i} r_{i} T}, \label{D0}
\end{eqnarray}
where $\sum_{i=1}^{n}{\mu_{i} r_{i} T}$ is the expected total downtime. Thus, if $D_{\gamma}$ is one, the targeted downtime $D_0$ is set to be equal to the expected total downtime.
There are 36 instances in the test bed $\Omega$. The setting of the fixed parameters in the test bed is given in Table \ref{tab:testbedmur}. $r_{i}$ varies for different $i$ by taking a value from {1, 3, 5}. $\mu_{i}$ varies for different $i$ as a sequence. Both lognormal distribution and uniform distribution are assumed as the distributions for the failure rates in this study.

%The number of the components $n$ in the system will affect the complexity, accuracy and computation time of our model. Different values of $\sigma_{i}$ reflect different deviation levels of the failure rate distributions. For the same sequence of $\mu_{i}$, higher $\sigma_{i}$ level results in higher exceeded total downtime. In our test beds we define $D_0$ as:
% \begin{eqnarray}
% D_{0}(\boldsymbol{x}) = D_{f}\sum_{i=1}^{n}{\sum_{j=1}^{m_{i}}{\mu_{ij}r_{ij}x_{ij}T}} \label{D0}
%\end{eqnarray}
%where $D_{f}$ is the level factor of $D_{0}(\boldsymbol{x})$, it reflects the intensity of the penalty policy service given by the customer within the service contract period $T$. Larger $D_{f}$ leads to small exceeded total downtime, furthermore, small penalty costs.

\begin{table}[htbp]
  \centering
  \caption{The parameter setting of the test bed}
    \begin{tabular}{lll}
    \toprule
  $n_{\alpha}$ & $\sigma_{\beta} (\forall i \in \{1,...,n_{\alpha}\})$ & $D_{\gamma}$\\
    \midrule
    5, 50, 100 & 0.2$\mu_{i}$, 0.35$\mu_{i}$, 0.5$\mu_{i}$ & $ 1, 1.1, 1.2, 1.3$ \\
  \bottomrule
    \end{tabular}%
  \label{tab:testbedps}%
\end{table}

\begin{table}[htbp]
  \centering
  \caption{Parameter values for $\mu_{i}$ and $r_{i}$, $i=\{1,2,...,n_{\alpha}\}$}
    \begin{tabular}{llll}
    \toprule
    $n_{\alpha}$ & $T$ & $\mu_{i}$ &  $r_{i}$\\
    \midrule
    $5$   &10& $\{0.500, 0.250, 0.167, 0.125, 0.100\}$ &$\{1,3,5,1,3\}$\\
    $50$  &10& $\{1,...,1/(\frac{1}{\mu_{i-1}}+0.182),...,0.101\}$ &$\{1,3,5,1,3,5,...,3\}$\\
    $100$ &10& $\{1,...,1/(\frac{1}{\mu_{i-1}}+0.091),...,0.100\}$ &$\{1,3,5,1,3,5,...,1\}$\\
    \bottomrule
    \end{tabular}%
  \label{tab:testbedmur}%
\end{table}

We present the results of the test bed in Table \ref{lognormal1}, \ref{lognormal2}, \ref{uniform1}, and \ref{uniform2}. To see how the approximation results deviate from the simulation results for each instance, we first compute the proportion of the expected exceeding downtime in the targeted total downtime, i.e., $D_{E}^{A}/D_0$ for the approximation model and $D_{E}^{S}/D_0$ for the simulation model, as well as the confidence intervals of the simulation results. The confidence intervals of all the simulation results are relatively small, which shows the accuracy of our simulation results. Then the accuracy of our approximation is assessed by the gaps between $D_{E}^{A}/D_0$ and $D_{E}^{S}/D_0$. For lognormally-distributed failure rates, the gaps are denoted as $G_l$ in Table \ref{lognormal1} and \ref{lognormal2}; the average gap is 0.0004 and the maximum gap is 0.0019. For uniformly-distributed failure rates, the gaps are denoted as $G_u$ in Table \ref{uniform1} and \ref{uniform2}; the average gap is  0.0005 and the maximum gap is 0.0023. From these results, we can get the conclusion that our approximation is relatively accurate under different parameter settings.

%which are denoted as $avg$ $ G_l$ and $max $ $G_l$ for lognormally-distributed failure rates, and denoted as $avg$ $G_n$ and $max$ $G_n$ for uniformly-distributed failure rates . The detailed results of $D_{E}^{A}/D_0$, $D_{E}^{S}/D_0$, confidence interval of the simulation result $D_{E}^{S}/D_0$, $avg$ $G$ and $max$ $G$ for each instance are shown in Table \ref{lognormal1} for lognormally-distributed failure rates and Table \ref{uniform1} for uniformly-distributed failure rates of \ref{detailapptable}. To take a better overview of the gaps, in Table \ref{generalapptable}, the comparison between $D_{E}^{A}/D_0$ and $D_{E}^{S}/D_0$ are made by computing $avg$ $G$ and $max$ $G$ for each level of every factor. For example, in the second row of Table \ref{generalapptable}, $avg$ $G$ and $max$ $G$ represent the average gap and maximum gap of the subset $\Omega_{n_{5}}=\{(n_{5},\sigma_{\beta},D_{\gamma})|\beta\in\{0.2,0.35,0.5\},\gamma\in\{01,02,03,04\}\}$ (when $n=5$).

Furthermore, our approximation method is more accurate when $n$ is large (e.g., 50 or 100). This is due to the fact that the total downtime is the sum of $n$ independent random variables, and when $n$ is large the total downtime converges in distribution to a normal random variable regardless of the individual underlying distributions. Note that the computation times of the approximation model for all the instances are negligible compared with the simulation model. In order to better demonstrate the accuracy of our approximation method, we also compare the probability density functions of the downtime distribution estimated from the simulation model with the ones estimated from the approximation model, as shown in Figure \ref{fig:lognormal} and Figure \ref{fig:uniform}. It is obvious that the downtime distributions estimated from the simulation model are approximately the same as the ones estimated from the approximation model for various settings of $n$ and $\sigma$.

%We can also observe that both $D_{E_{p}}^{S}$ and $D_{E_{p}}^{A}$ is much lower for a system with a larger number of components than for a system with small number of components. And $D_{E_{p}}$ is increasing while $\sigma$ is increasing.
\begin{table}[htbp]
  \centering
  \caption{Simulation and approximation results in the case of lognormal distributed failure rates}
    \begin{tabular}{rrrcr}
    \toprule
          & $D_{E}^{A}/D_0$ & $D_{E}^{S}/D_0$ & Confidence interval of $D_{E}^{S}/D_0$ &  $G_{l}$ \\
    \midrule
    $(n_{5},\sigma_{0.2},D_{01})$ & 14,58 & 14,70 & (14,68, 14,72) & 0,0012 \\
    $(n_{5},\sigma_{0.2},D_{02})$ & 9,62  & 9,59  & (9,57, 9,60) & 0,0003 \\
    $(n_{5},\sigma_{0.2},D_{03})$ & 6,26  & 6,11  & (6,10, 6,13) & 0,0015 \\
    $(n_{5},\sigma_{0.2},D_{04})$ & 4,02  & 3,84  & (3,83, 3,85) & 0,0019 \\
    $(n_{5},\sigma_{0.35},D_{01})$ & 15,65 & 15,73 & (15,70, 15,75) & 0,0008 \\
    $(n_{5},\sigma_{0.35},D_{02})$ & 10,58 & 10,55 & (10,53, 10,56) & 0,0004 \\
    $(n_{5},\sigma_{0.35},D_{03})$& 7,09  & 6,97  & (6,95, 6,98) & 0,0013 \\
    $(n_{5},\sigma_{0.35},D_{04})$ & 4,71  & 4,55  & (4,53, 4,56) & 0,0016 \\
    $(n_{5},\sigma_{0.5},D_{01})$ & 17,15 & 17,15 & (17,13, 17,17) & 0,0000 \\
    $(n_{5},\sigma_{0.5},D_{02})$ & 11,96 & 11,89 & (11,87, 11,91) & 0,0006 \\
    $(n_{5},\sigma_{0.5},D_{03})$ & 8,29  & 8,18  & (8,16, 8,20) & 0,0011 \\
    $(n_{5},\sigma_{0.5},D_{04})$ & 5,72  & 5,60  & (5,58, 5,61) & 0,0012 \\
    $(n_{50},\sigma_{0.2},D_{01})$ & 4,28  & 4,29  & (4,28, 4,29) & 0,0001 \\
    $(n_{50},\sigma_{0.2},D_{02})$ & 1,00  & 0,98  & (0,98, 0,98) & 0,0003 \\
    $(n_{50},\sigma_{0.2},D_{03})$ & 0,15  & 0,14  & (0,14, 0,14) & 0,0001 \\
    $(n_{50},\sigma_{0.2},D_{04})$ & 0,02  & 0,01  & (0,01, 0,01) & 0,0000 \\
    $(n_{50},\sigma_{0.35},D_{01})$ & 4,82  & 4,82  & (4,81, 4,83) & 0,0000 \\
    $(n_{50},\sigma_{0.35},D_{02})$ & 1,36  & 1,34  & (1,34, 1,35) & 0,0002 \\
    $(n_{50},\sigma_{0.35},D_{03})$& 0,28  & 0,27  & (0,27, 0,27) & 0,0001 \\
    $(n_{50},\sigma_{0.35},D_{04})$ & 0,04  & 0,04  & (0,04, 0,04) & 0,0000 \\
    $(n_{50},\sigma_{0.5},D_{01})$ & 5,56  & 5,54  & (5,53, 5,55) & 0,0002 \\
    $(n_{50},\sigma_{0.5},D_{02})$& 1,88  & 1,89  & (1,89, 1,90) & 0,0001 \\
    $(n_{50},\sigma_{0.5},D_{03})$ & 0,51  & 0,52  & (0,52, 0,52) & 0,0002 \\
    $(n_{50},\sigma_{0.5},D_{04})$ & 0,11  & 0,12  & (0,12, 0,12) & 0,0001 \\
   $(n_{100},\sigma_{0.2},D_{01})$ & 3,05  & 3,05  & (3,04, 3,05) & 0,0000 \\
   $(n_{100},\sigma_{0.2},D_{02})$  & 0,35  & 0,34  & (0,33, 0,34) & 0,0001 \\
   $(n_{100},\sigma_{0.2},D_{03})$  & 0,02  & 0,01  & (0,01, 0,01) & 0,0000 \\
   $(n_{100},\sigma_{0.2},D_{04})$  & 0,00  & 0,00  & (0,00, 0,00) & 0,0000 \\
   $(n_{100},\sigma_{0.35},D_{01})$ & 3,44  & 3,44  & (3,43, 3,44) & 0,0000 \\
   $(n_{100},\sigma_{0.35},D_{02})$ & 0,53  & 0,52  & (0,52, 0,53) & 0,0000 \\
   $(n_{100},\sigma_{0.35},D_{03})$& 0,04  & 0,04  & (0,04, 0,04) & 0,0000 \\
   $(n_{100},\sigma_{0.35},D_{01})$ & 0,00  & 0,00  & (0,00, 0,00) & 0,0000 \\
    \bottomrule
    \end{tabular}%
  \label{lognormal1}%
\end{table}%

\begin{table}[htbp]
  \centering
  \caption{(Continued) Simulation and approximation results in the case of lognormal distributed failure rates}
    \begin{tabular}{rrrcrr}
    \toprule
          & $D_{E}^{A}/D_0$ & $D_{E}^{S}/D_0$ & Confidence interval of $D_{E}^{S}/D_0$ &  $G_{l}$ \\
    \midrule
 $(n_{100},\sigma_{0.5},D_{01})$  & 3,97  & 3,96  & (3,96, 3,97) & 0,0001 \\
   $(n_{100},\sigma_{0.5},D_{02})$ & 0,82  & 0,83  & (0,82, 0,83) & 0,0001 \\
   $(n_{100},\sigma_{0.5},D_{03})$ & 0,10  & 0,11  & (0,11, 0,11) & 0,0001 \\
   $(n_{100},\sigma_{0.5},D_{04})$ & 0,01  & 0,01  & (0,01, 0,01) & 0,0000 \\
      \bottomrule
     \end{tabular}%
  \label{lognormal2}%
\end{table}%

% Table generated by Excel2LaTeX from sheet 'erlanguniformS100000'
\begin{table}[htbp]
  \centering
  \caption{Simulation and approximation results in the case of uniform distributed failure rates}
 \begin{tabular}{rrrcr}
    \toprule
          & $D_{E}^{A}/D_0$ & $D_{E}^{S}/D_0$ & Confidence interval of $D_{E}^{S}/D_0$ &  $G_{u}$ \\
    \midrule
    $(n_{5},\sigma_{0.2},D_{01})$ & 14,58 & 14,70 & (14,68, 14,7) & 0,0012 \\
    $(n_{5},\sigma_{0.2},D_{02})$  & 9,62  & 9,60  & (9,58, 9,6) & 0,0002 \\
    $(n_{5},\sigma_{0.2},D_{03})$  & 6,26  & 6,12  & (6,1, 6,12) & 0,0014 \\
    $(n_{5},\sigma_{0.2},D_{04})$  & 4,02  & 3,83  & (3,82, 3,83) & 0,0019 \\
    $(n_{5},\sigma_{0.35},D_{01})$ & 15,65 & 15,82 & (15,8, 15,82) & 0,0017 \\
    $(n_{5},\sigma_{0.35},D_{02})$ & 10,58 & 10,57 & (10,55, 10,57) & 0,0001 \\
    $(n_{5},\sigma_{0.35},D_{03})$ & 7,09  & 6,97  & (6,96, 6,97) & 0,0012 \\
    $(n_{5},\sigma_{0.35},D_{04})$ & 4,71  & 4,53  & (4,52, 4,53) & 0,0018 \\
    $(n_{5},\sigma_{0.5},D_{01})$ & 17,15 & 17,38 & (17,35, 17,38) & 0,0023 \\
    $(n_{5},\sigma_{0.5},D_{02})$ & 11,96 & 12,01 & (11,99, 12,01) & 0,0005 \\
    $(n_{5},\sigma_{0.5},D_{03})$ & 8,29  & 8,18  & (8,17, 8,18) & 0,0011 \\
    $(n_{5},\sigma_{0.5},D_{04})$ & 5,72  & 5,54  & (5,53, 5,54) & 0,0018\\
    $(n_{50},\sigma_{0.2},D_{01})$  & 4,28  & 4,28  & (4,28, 4,28) & 0,0001 \\
    $(n_{50},\sigma_{0.2},D_{02})$ & 1,00  & 0,98  & (0,97, 0,98) & 0,0003 \\
    $(n_{50},\sigma_{0.2},D_{03})$  & 0,15  & 0,13  & (0,13, 0,13) & 0,0002 \\
    $(n_{50},\sigma_{0.2},D_{04})$  & 0,02  & 0,01  & (0,01, 0,01) & 0,0000 \\
    $(n_{50},\sigma_{0.35},D_{01})$ & 4,82  & 4,83  & (4,83, 4,83) & 0,0001 \\
    $(n_{50},\sigma_{0.35},D_{02})$ & 1,36  & 1,33  & (1,33, 1,33) & 0,0003 \\
    $(n_{50},\sigma_{0.35},D_{03})$ & 0,28  & 0,25  & (0,25, 0,25) & 0,0003 \\
    $(n_{50},\sigma_{0.35},D_{04})$ & 0,04  & 0,03  & (0,03, 0,03) & 0,0001\\
    $(n_{50},\sigma_{0.5},D_{01})$ & 5,56  & 5,58  & (5,57, 5,58) & 0,0002 \\
    $(n_{50},\sigma_{0.5},D_{02})$  & 1,88  & 1,84  & (1,83, 1,84) & 0,0005 \\
    $(n_{50},\sigma_{0.5},D_{03})$ & 0,51  & 0,46  & (0,45, 0,46) & 0,0005 \\
    $(n_{50},\sigma_{0.5},D_{04})$ & 0,11  & 0,08  & (0,08, 0,08) & 0,0002 \\
    $(n_{100},\sigma_{0.2},D_{01})$  & 3,05  & 3,04  & (3,04, 3,04) & 0,0000 \\
    $(n_{100},\sigma_{0.2},D_{02})$ & 0,35  & 0,33  & (0,33, 0,33) & 0,0002 \\
    $(n_{100},\sigma_{0.2},D_{03})$ & 0,02  & 0,01  & (0,01, 0,01) & 0,0000 \\
    $(n_{100},\sigma_{0.2},D_{04})$ & 0,00  & 0,00  & (0,00, 0,00) & 0,0000 \\
    $(n_{100},\sigma_{0.35},D_{01})$ & 3,44  & 3,44  & (3,44, 3,44) & 0,0001 \\
    $(n_{100},\sigma_{0.35},D_{02})$ & 0,53  & 0,51  & (0,51, 0,51) & 0,0002 \\
    $(n_{100},\sigma_{0.35},D_{03})$ & 0,04  & 0,03  & (0,03, 0,03) & 0,0001 \\
    $(n_{100},\sigma_{0.35},D_{04})$ & 0,00  & 0,00  & (0,00, 0,00) & 0,0000 \\
    \bottomrule
    \end{tabular}%
  \label{uniform1}%
\end{table}%

\begin{table}[htbp]
  \centering
  \caption{(Continued) Simulation and approximation results in the case of uniform distributed failure rates}
    \begin{tabular}{rrrrr}
     \toprule
          & $D_{E}^{A}/D_0$ & $D_{E}^{S}/D_0$ & Confidence interval of $D_{E}^{S}/D_0$ &  $G_{u}$ \\
    \midrule
    $(n_{100},\sigma_{0.5},D_{01})$  & 3,97  & 3,98  & (3,97, 3,98) & 0,0001 \\
    $(n_{100},\sigma_{0.5},D_{02})$ & 0,82  & 0,79  & (0,79, 0,79) & 0,0003 \\
    $(n_{100},\sigma_{0.5},D_{03})$ & 0,10  & 0,09  & (0,08, 0,09) & 0,0001\\
    $(n_{100},\sigma_{0.5},D_{04})$ & 0,01  & 0,00  & (0,00, 0,00) & 0,0000 \\
    \bottomrule
    \end{tabular}%
  \label{uniform2}%
\end{table}%


\section{Optimization}

In the previous section, we have discussed how a given policy, which is the selection of rough designs for all the critical components in the system can be evaluated in an approximate way. In this section, we will describe a method of finding a feasible policy for problem (P) with as low life cycle costs as possible. The problem is formulated as a binary integer programming with a nonlinear objective function.
%which is equivalent to a knapsack problem with multiple-choice constraints, so that is $NP$-hard (Garey and Johnson 1979).

According to Eq. \eqref{EXD1}, the expected penalty costs $\mathbb{E}_{\Lambda(\boldsymbol(X))}[P(\boldsymbol(X))]$ of problem (P) can be written as:
\begin{eqnarray}
\mathbb{E}_{\Lambda(\boldsymbol{x})}[P_{A}(\boldsymbol{x})]&=&c_{p}(\frac{k(\boldsymbol{x})-q(\boldsymbol{x})}{\theta(\boldsymbol{x})}-D_{0})\sum_{j=0}^{k(\boldsymbol{x})-2}{\frac{(\theta(\boldsymbol{x}) D_{0})^j}{j!}e^{-\theta(\boldsymbol{x}) D_{0}}}\nonumber\\
&&+c_{p}(\frac{k(\boldsymbol{x})-q(\boldsymbol{x})}{\theta(\boldsymbol{x})})\frac{(\theta(\boldsymbol{x})D_{0})^{k(\boldsymbol{x})-1}}{(k(\boldsymbol{x})-1)!}e^{-\theta(\boldsymbol{x})D_{0}}
\end{eqnarray}
The we have the optimization problem formulated as:
\begin{eqnarray}
\text{($P_{A}$)} \hspace{15mm} & min_{\boldsymbol{x}} & \hspace{10mm} \mathbb{E}_{\Lambda} \bigg[ \pi_{A}(\boldsymbol{x})\bigg] \nonumber\\
& \text{ s.t. }&  \hspace{5mm} \sum_{j=1}^{m_i}{x_{ij}}=1, \hspace{10mm} \forall i \in \{1,2,...,n\} \nonumber
\end{eqnarray}
where $\pi_{A}(\boldsymbol{x})=A(\boldsymbol{x})+R(\boldsymbol{x})+P_{A}(\boldsymbol{x})$.


%The branch-and-bound (B\&B) method is the most widely used algorithm for solving Integer programming problems. However, according to Taha 1975, the original (B\&B) method in the Land-Doig method is not suitable to solve a non-linear integer programming (NIP) problem, because the validity of the branching rules requires linearity assumption. Dakin 1965 proposed a modified Land-Doig method proposed by makes the branching rule suitable for solving non-linear integer programming problem, under the condition that the objective function and each constraint function with respect to decision variables are concave and convex, respectively. Lee \textit {et al}. 1994 presented a B\&B method for solving separable convex integer programming problems. Sung and Cho 2000 have solved a reliability optimization problem of designing a hardware system involving numerous discrete choices among available component types among reliability, cost, performance, weight, etc. using an efficient B\&B problem. Since Sung and Cho 2000 generated upper bounds and lower bounds for each stage of their problem, the objective function should be decomposable. Another method for solving NIP problems is intelligent enumeration proposed by Taha 1975. Taha's method can only solve polynomial IP problems. For large reality problems, exact methods are lacking since they require a very large amount of computation time and specific properties of the objective function and constraints.
\subsection{Exact analysis}
\subsection{Numerical example}
% Table generated by Excel2LaTeX from sheet 'Sheet1'
\begin{table}[htbp]
  \centering
  \caption{Parameter explanation of the numerical example}
    \begin{tabular}{ll}
    \toprule
    Parameter & Explanation \\
    \midrule
    $c_{a}^{ij}$ & acquistion costs of each desgin \\
    $c_{r}^{ij}=c_{a}^{ij}*\{5\%, 10\%, 20\%\}$ & repair costs of each design  \\
    $r_{ij}$ & repair time, randomly generated by a $U(2,48)$ distribution  \\
    $\mu_{ij}$ & mean of the failure rate, $\mu_{i1}$ are randomly generated\\
     & from $U(0.1, 0.3)$, $\mu_{2i}$ are randomly generated from $U(0.4, 0.6)$ \\
    $\sigma_{ij}$ & $\sigma_{i1}=\mu_{i1}*\{20\%, 35\%, 50\%\}$, $\sigma_{i2}=\mu_{i2}*\{35\%, 50\%, 20\%\}$ \\
    \bottomrule
    \end{tabular}%
  \label{tab:addlabel}%
\end{table}%

% Table generated by Excel2LaTeX from sheet 'Sheet1'
\begin{table}[htbp]
  \centering
  \caption{Parameter design of the numberical example}
    \begin{tabular}{rrrrrrrrrrr}
    \toprule
Comp.          & \multicolumn{5}{c}{Alternative design 1} & \multicolumn{5}{c}{Alternative design 2} \\
Number    & $c_{a}^{i1}$ & $c_{r}^{i1}$ & $r_{i1}$ & $\mu_{i1}$ & $\sigma_{i1}$ & $c_{a}^{i2}$ & $c_{r}^{i2}$ & $r_{i2}$ & $\mu_{i2}$ & $\sigma_{i2}$ \\
    \midrule
    1     & 10000 & 500   & 12    & 0,23  & 0,05  & 5000  & 250   & 32    & 0,47  & 0,17 \\
    2     & 9675  & 967,5 & 32    & 0,17  & 0,06  & 4837,5 & 483,75 & 25    & 0,41  & 0,20 \\
    3     & 9350  & 1870  & 28    & 0,15  & 0,07  & 4675  & 935   & 16    & 0,45  & 0,09 \\
    4     & 9025  & 451,25 & 10    & 0,25  & 0,05  & 4512,5 & 225,625 & 35    & 0,60  & 0,21 \\
    5     & 8700  & 870   & 41    & 0,14  & 0,05  & 4350  & 435   & 14    & 0,46  & 0,23 \\
    6     & 8375  & 1675  & 5     & 0,25  & 0,12  & 4187,5 & 837,5 & 25    & 0,51  & 0,10 \\
    7     & 8050  & 402,5 & 11    & 0,21  & 0,04  & 4025  & 201,25 & 27    & 0,53  & 0,18 \\
    8     & 7725  & 772,5 & 12    & 0,24  & 0,08  & 3862,5 & 386,25 & 38    & 0,55  & 0,27 \\
    9     & 7400  & 1480  & 12    & 0,14  & 0,07  & 3700  & 740   & 2     & 0,54  & 0,11 \\
    10    & 7075  & 353,75 & 8     & 0,29  & 0,06  & 3537,5 & 176,875 & 23    & 0,52  & 0,18 \\
    11    & 6750  & 675   & 20    & 0,14  & 0,05  & 3375  & 337,5 & 15    & 0,55  & 0,27 \\
    12    & 6425  & 1285  & 34    & 0,21  & 0,11  & 3212,5 & 642,5 & 4     & 0,46  & 0,09 \\
    13    & 6100  & 305   & 31    & 0,20  & 0,04  & 3050  & 152,5 & 9     & 0,53  & 0,18 \\
    14    & 5775  & 577,5 & 8     & 0,26  & 0,09  & 2887,5 & 288,75 & 21    & 0,57  & 0,29 \\
    15    & 5450  & 1090  & 16    & 0,29  & 0,14  & 2725  & 545   & 10    & 0,58  & 0,12 \\
    16    & 5125  & 256,25 & 41    & 0,16  & 0,03  & 2562,5 & 128,125 & 15    & 0,42  & 0,15 \\
    17    & 4800  & 480   & 47    & 0,18  & 0,06  & 2400  & 240   & 17    & 0,44  & 0,22 \\
    18    & 4475  & 895   & 33    & 0,27  & 0,13  & 2237,5 & 447,5 & 27    & 0,51  & 0,10 \\
    19    & 4150  & 207,5 & 9     & 0,16  & 0,03  & 2075  & 103,75 & 27    & 0,50  & 0,18 \\
    20    & 3825  & 382,5 & 19    & 0,20  & 0,07  & 1912,5 & 191,25 & 33    & 0,45  & 0,23 \\
    21    & 3500  & 700   & 33    & 0,25  & 0,12  & 1750  & 350   & 9     & 0,54  & 0,11 \\
    22    & 3175  & 158,75 & 46    & 0,12  & 0,02  & 1587,5 & 79,375 & 30    & 0,43  & 0,15 \\
    23    & 2850  & 285   & 5     & 0,11  & 0,04  & 1425  & 142,5 & 19    & 0,48  & 0,24 \\
    24    & 2525  & 505   & 37    & 0,13  & 0,07  & 1262,5 & 252,5 & 30    & 0,44  & 0,09 \\
    25    & 2200  & 110   & 21    & 0,28  & 0,06  & 1100  & 55    & 9     & 0,53  & 0,19 \\
    26    & 1875  & 187,5 & 6     & 0,19  & 0,07  & 937,5 & 93,75 & 37    & 0,57  & 0,28 \\
    27    & 1550  & 310   & 34    & 0,14  & 0,07  & 775   & 155   & 26    & 0,42  & 0,08 \\
    \bottomrule
    \end{tabular}%
  \label{tab:addlabel}%
\end{table}%


\begin{table}[htbp]
  \centering
  \caption{Optimization results of the numerical example for problem $P_{A}$ by using approximation evaluation and complete enumeration}
    \begin{tabular}{rrrll}
    \toprule
    n     & $D_{0}$  &$\mathbb{E}_{\Lambda} \bigg[ \pi_{A}(\boldsymbol{x_{A}})\bigg]$  & $\boldsymbol{x_{A}}$  & CPU time(s) \\
    \midrule
    2     & 167,45 & 24882 & 11    & 0 \\
    3     & 224,45 & 37013 & 112   & 0 \\
    4     & 341,95 & 42103 & 1121  & 0 \\
    5     & 402,85 & 49398 & 11212 & 0 \\
    6     & 472,85 & 58924 & 122121 & 0 \\
    7     & 555,95 & 65847 & 1221211 & 0 \\
    8     & 674,85 & 73646 & 12212211 & 0 \\
    9     & 688,65 & 81178 & 122122112 & 0 \\
    10    & 760,05 & 87580 & 1221221121 & 0 \\
    11    & 815,3 & 94690 & 12212111222 & 0 \\
    12    & 860,2 & 99836 & 122122112122 & 0 \\
    13    & 915,05 & 103603 & 1221221121222 & 1 \\
    14    & 985,3 & 109704 & 12212211212221 & 2 \\
    15    & 1037,5 & 115784 & 122122112122212 & 4 \\
    16    & 1101,8 & 119121 & 1221221121222122 & 8 \\
    17    & 1181,5 & 123020 & 12212211212221222 & 16 \\
    18    & 1294,9 & 129466 & 122121112222212222 & 32 \\
    19    & 1369,6 & 131993 & 1221221121222122221 & 64 \\
    20    & 1462,85 & 135476 & 22212111212221222211 & 129 \\
    21    & 1528,4 & 138689 & 122122112222212222112 & 261 \\
    22    & 1620,5 & 141239 & 1221221122222122221121 & 526 \\
    23    & 1668,85 & 143169 & 12212211222221222211211 & 1062 \\
    24    & 1758,9 & 145565 & 222122112122212222112111 & 2119 \\
    25    & 1812,15 & 146897 & 2221221121222122221121112 & 4330 \\
    26    & 1923,3 & 146089 & 22212211222221222211211121 & 8732 \\
    27    & 2001,7 & 147558 & 222122112222212222112111211 & 17485 \\
    \bottomrule
    \end{tabular}%
  \label{tab:addlabel}%
\end{table}%

\begin{table}[htbp]
  \centering
  \caption{Optimization results from problem $P_{A}$ by using simulation evaluation and complete enumeration}
    \begin{tabular}{rrrrlrr}
    \toprule
    n     & $\mathbb{E}_{\Lambda} \bigg[ \pi_{A}(\boldsymbol{x_{A}})\bigg]$ & $\mathbb{E}_{\Lambda} \bigg[ \pi_{S}(\boldsymbol{x})\bigg]$ & $\mathbb{E}_{\Lambda} \bigg[ \pi_{S}(\boldsymbol{x_{A}})\bigg]$ & $\boldsymbol{x_{S}}$ & Gap1  & Gap2 \\
    \midrule
    2     & 24881,8 & 24445,5 & -     & 11    & 1,78\% & - \\
    3     & 37013,3 & 36483,0 & -     & 112   & 1,45\% & - \\
    4     & 42103,3 & 42028,9 & -     & 1121  & 0,18\% & - \\
    5     & 49397,5 & 49312,0 & -     & 11212 & 0,17\% & - \\
    6     & 58924,0 & 59195,4 & -     & 122121 & 0,46\% & - \\
    7     & 65847,4 & 66079,0 & -     & 1221211 & 0,35\% & - \\
    8     & 73645,5 & 73706,5 & -     & 12212211 & 0,08\% & - \\
    9     & 81177,6 & 81165,8 & -     & 122122112 & 0,01\% & - \\
    10    & 87580,5 & 87673,1 & -     & 1221221121 & 0,11\% & - \\
    11    & 94689,9 & 94795,8 & -     & 12212111222 & 0,11\% & - \\
    12    & 99835,9 & 99775,1 & -     & 122122112122 & 0,06\% & - \\
    13    & 103603,2 & 103771,8 & 103802,3 & 1221211122222 & -     & 0,19\% \\
    14    & 109703,6 & 109886,1 & -     & 12212211212221 & 0,17\% & - \\
    15    & 115784,0 & 115828,2 & 116066,5 & 122122112212212 & -     & 0,24\% \\
    16    & 119120,9 & 119028,3 & -     & 1221221121222122 & 0,08\% & - \\
    17    & 123020,0 & 123323,3 & 123192,0 & 12212111222221222 & -     & 0,14\% \\
    18    & 129466,4 & 129350,8 & -     & 122121112222212222 & 0,09\% & - \\
    19    & 131992,9 & 131789,2 & -     & 1221221121222122221 & 0,15\% & - \\
    20    & 135476,0 & 135478,0 & 135616,2 & 12212211212221222211 & -     & 0,10\% \\
    21    & 138688,7 & 138622,9 & -     & 122122112222212222112 & 0,05\% & - \\
    \bottomrule
    \end{tabular}%
  \label{tab:stosimopti}%
\end{table}%


% Table generated by Excel2LaTeX from sheet 'Sheet2'
\begin{table}[htbp]
  \indent
  \caption{Optimization results for deterministic model setting }
    \begin{tabular}{rrrlrrr}
    \toprule
    $n$     & $D_{0}$  & $\pi(\boldsymbol{x}_{D})$ & $\boldsymbol{x}_{D}$ & $\mathbb{E}_{\Lambda} \bigg[ \pi(\boldsymbol{x}_{D})\bigg]$ & $\Delta$ & $\% \Delta$ \\
    \midrule
    2     & 167,45 & 17971 & 12    & 34048 & 9166  & 26,92\% \\
    3     & 224,45 & 26853 & 122   & 50456 & 13443 & 26,64\% \\
    4     & 341,95 & 35304 & 2211  & 73908 & 31804 & 43,03\% \\
    5     & 402,85 & 41655 & 22112 & 84661 & 35263 & 41,65\% \\
    6     & 472,85 & 50945 & 222121 & 83758 & 24834 & 29,65\% \\
    7     & 555,95 & 59009 & 2211221 & 106884 & 41037 & 38,39\% \\
    8     & 674,85 & 65132 & 22222111 & 128448 & 54803 & 42,67\% \\
    9     & 688,65 & 72828 & 222221112 & 134873 & 53695 & 39,81\% \\
    10    & 760,05 & 77309 & 2221222121 & 137112 & 49532 & 36,12\% \\
    11    & 815,30 & 82700 & 22212211222 & 145550 & 50860 & 34,94\% \\
    12    & 860,20 & 88708 & 222122212122 & 152331 & 52495 & 34,46\% \\
    13    & 915,05 & 92566 & 2221222121222 & 153838 & 50235 & 32,65\% \\
    14    & 985,30 & 98663 & 22212221221221  & 159213 & 49509 & 31,10\% \\
    15    & 1037,50 & 104549 & 222122212212212  & 168340 & 52556 & 31,22\% \\
    16    & 1101,80 & 107650 & 2221222122122122  & 172421 & 53300 & 30,91\% \\
    17    & 1181,50 & 111106 & 22212221221221222  & 176535 & 53515 & 30,31\% \\
    18    & 1294,90 & 115625 & 222122212212212222  & 196383 & 66917 & 34,07\% \\
    19    & 1369,60 & 117644 & 2221222122222122221 & 197933 & 65940 & 33,31\% \\
    20    & 1462,85 & 121862 & 22212221222222222111  & 206425 & 70949 & 34,37\% \\
    21    & 1528,40 & 125390 & 222222112222212222112  & 209560 & 70871 & 33,82\% \\
    22    & 1620,50 & 126496 & 2221222122222222221121  & 211114 & 69875 & 33,10\% \\
    23    & 1668,85 & 129659 & 22212221222222222211211  & 196555 & 53386 & 27,16\% \\
    24    & 1758,90 & 131297 & 222222212222212222112111  & 223816 & 78251 & 34,96\% \\
    25    & 1812,15 & 132689 & 2222222122222122221121112 & 223276 & 76378 & 34,21\% \\
    26    & 1923,30 & 132177 & 22222221222222222211211121  & 227036 & 80947 & 35,65\% \\
    27    & 2001,70 & 134161 & 222222212222222222112111211 & 217550 & 69992 & 32,17\% \\
    \bottomrule
    \end{tabular}%
  \label{tab:deteroptimization}%
\end{table}%


\section{Conclusion}

\begin{appendices}


\section{Procedures of the Monte Carlo simulation}
\label{MCP}
\begin{description}
\item[Step 1]
First, we generated the sequences of $r_{i}$, $\mu_{i}$ and $\sigma_{i}$, $i=\{1,2,...,n\}$. Then we get $D_0$ immediately from Eq. \eqref{D0}. Furthermore, we take one sample $\hat{\Lambda}_{i}$ from $\Lambda_{i} \sim G(\mu_{i},\sigma_{i})$ for each component to simulate its failure rate $\lambda_{i}$, where $G(\mu_{i},\sigma_{i})$ is a general distribution with parameter mean $\mu_{i}$ and standard deviation $\sigma_{i}$. Given that the number of failures of each component $s_{i}$ is Poisson distributed with parameter $\lambda_{i}T$, we take one sample $\hat{S}_{i}$ from $S_{i}\sim$Pois$(\hat{\Lambda}_{i}T)$. Together with $r_{i}$, we get one simulation result $\hat{D}_{E}^{S}/D_{0}$ of the proportion of the exceeded total downtime $D_{E}/D_0$ computed as following:
\begin{eqnarray}
\hat{D}_{E}^{S}/D_{0}=\frac{\sum_{i=1}^{n}{\hat{S}_{i}r_{i}}-D_0}{\sum_{i=1}^{n}{\hat{S}_{i}r_{i}}}
\end{eqnarray}

\item[Step 2]

Repeat step 1 for 10000 times to get 10000 $\hat{D}_{E}^{S}/D_{0}$, we take the expected value $\bar{D}_{E}^{S}/D_{0}$ of all the $\hat{D}_{E}^{S}/D_{0}$ and compare it with the approximation result $D_{E}^{A}/D_{0}$ according to Eqs. \eqref{EXD1} or \eqref{EXD2}  to get a value of the absolute gap between $\bar{D}_{E}^{S}/D_{0}$ and $D_{E}^{A}/D_{0}$.

\item[Step 3]

Repeat step 3 for 50 times to generate final simulation results of $D_{E}/D_{0}$, $D_{E}^{S}/D_{0}$. And the mean value $avg$ $G$, maximum value $max$ $G$ and the confidence interval for $D_{E_{p}}^{S}$ of the gap between $D_{E}^{S}/D_{0}$ and $D_{E}^{A}/D_{0}$. The 95\% percent confidence interval is given as:
$$(D_{E}^{S}/D_{0}-t_{(49,2.5\%)}\sqrt{\frac{S^{2}(50)}{50}},\hspace{3mm}D_{E}^{S}/D_{0}+t_{(49,2.5\%)}\sqrt{\frac{S^{2}(50)}{50}} ) $$

\end{description}

\section{Derivations for the two-moment fits method}
\label{AG}
\subsection{The first two moments of the downtime distribution}
The mean $\mu_{D}(\boldsymbol{x})$ of $\mathbb{E}_{S}\bigg[D(\boldsymbol{x})\bigg]$ are given by:
\begin{eqnarray}
\mu_{D}(\boldsymbol{x})=\mathbb{E}_{\Lambda(\boldsymbol{x})}\mathbb{E}_{S} \bigg[D(\boldsymbol{x}) \bigg]=T\sum_{i=1}^{n}{r_{i}(\boldsymbol{x_i})\mu_{i}(\boldsymbol{x_i})}=\frac{k(\boldsymbol{x})-q(\boldsymbol{x})}{\theta(\boldsymbol{x})}
\label{muD}
\end{eqnarray}
The variance $\sigma_{D}(\boldsymbol{x})$ of $D(\boldsymbol{x})$ is given by:
\begin{eqnarray}
\sigma^{2}_{D}(\boldsymbol{x})&=&\mathbb{E}_{\Lambda(\boldsymbol{x})}\mathbb{E}_{S} \bigg[D^{2}(\boldsymbol{x}) \bigg]- \bigg\{\mathbb{E}_{\Lambda(\boldsymbol{x})}\mathbb{E}_{S}\bigg[D(\boldsymbol{x}) \bigg] \bigg\}^{2} \nonumber\\
&=& \mathbb{E}_{\Lambda(\boldsymbol{x})}\bigg[\bigg[\sum_{i=1}^{n}{r_{i}(\boldsymbol{x_{i}})\lambda_{i}(\boldsymbol{x_{i}})T}\bigg]^{2} +\sum_{i=1}^{n}{r_{i}^{2}(\boldsymbol{x_{i}})\lambda_{i}(\boldsymbol{x_{i}})T}\bigg]-\bigg[T\sum_{i=1}^{n}{r_{i}(\boldsymbol{x_i})\mu_{i}(\boldsymbol{x_i})}\bigg]^{2}\nonumber\\
&=&\sum_{i=1}^{n}{r_{i}^{2}(\boldsymbol{x_i})\mu_{i}(\boldsymbol{x_i})T}\nonumber\\
&=&\frac{\sqrt{k(\boldsymbol{x})-q^{2}(\boldsymbol{x})}}{\lambda(\boldsymbol{x})}
\end{eqnarray}
\subsection{Identities for the Erlang$(k-1,k)$ distribution}
Consider an Erlang$(k-1,k)$ distribution $X$ with parameters $(k,\theta,q)$, the probability density function of is
\begin{eqnarray}
e_{k-1,k}(x)=q\theta^{k-1}\frac{x^{k-2}}{(k-2)!}e^{-\theta x} + (1-q)\theta^{k}\frac{x^{k-1}}{(k-1)!}e^{-\theta x} \nonumber
\end{eqnarray}
the cumulative distribution function of $X$ is
\begin{eqnarray}
E_{k-1,k}(x)=q\bigg(1-\sum_{j=0}^{k-2}\frac{(\theta x)^{j}}{j!}e^{-\theta x} \bigg)+(1-q)\bigg(1-\sum_{j=0}^{k-1}\frac{(\theta x)^{j}}{j!}e^{-\theta x} \bigg) \nonumber
\end{eqnarray}
Then the first partial moment of the Erlang$(k-1,k)$ can be described as:
\begin{eqnarray}
\mathbb{E}{[(X-X_{0})^{+}]} &=& \int_{0}^{\infty}{(x-X_{0})^{+}e_{k-1,k}(x)d x} \nonumber\\
&=& \frac{q(k-1)}{\theta}\int_{X_{0}}^{\infty}{\theta^{k}\frac{x^{k-1}}{(k-1)!}e^{-\theta x}dx}+\frac{k(1-q)}{\theta}\int_{X_0}^{\infty}{\theta^{k+1}\frac{x^{k}}{k!}e^{-\theta x}dx}\nonumber\\
&&-X_{0}\int_{X_{0}}^{\infty}{e_{k-1,k}(x)dx}\nonumber
\end{eqnarray}
Given the probability density function of an Erlang distribution is: $e_{k}^{\theta}(x)=\theta^{k}\frac{x^{k-1}}{(k-1)!}e^{-\theta x}$ and the Erlang cumulative distribution function is $E_{k}^{\theta}(X) = 1-\sum_{j=0}^{k-1}{\frac{(\lambda x)^{j}}{j!}e^{-\lambda x}}$, we have:
\begin{eqnarray}
\mathbb{E}{[(X-X_{0})^{+}]}
&=& \frac{q(k-1)}{\theta}\bigg[1-E_{k}^{\theta}(X_{0})\bigg]+\frac{k(1-q)}{\theta}\bigg[1-E_{k+1}^{\theta}(X_{0})\bigg]-X_{0}\bigg[1-E_{k-1,k}^{\theta}(X_{0})\bigg] \nonumber\\
&=&(\frac{k-q}{\theta}-X_{0})\sum_{j=0}^{k-2}{\frac{(\theta X_{0})^j}{j!}e^{-\theta X_{0}}}+(\frac{k-q}{\theta})\frac{(\theta X_{0})^{k-1}}{(k-1)!}e^{-\theta X_{0}}\nonumber
\end{eqnarray}


\subsection{Identities for the Hyperexponential distribution}
For a Hyperexponential distribution $X$ with parameters $(\theta_1,\theta_2,q)$, the probability density function is given as:
\begin{eqnarray}
h_{2}(x) = q\theta_{1}e^{-\theta_{1}x} + (1-q)\theta_{2}e^{-\theta_{2}x} \nonumber
\end{eqnarray}
the cumulative distribution function is given as:
\begin{eqnarray}
H_{2}(x) = q(1-e^{-\theta_{1}x})+(1-q)(1-e^{-\theta_{2}x}) \nonumber
\end{eqnarray}
And the first partial moment is given as:
\begin{eqnarray}
\mathbb{E}{[(X-X_{0})^{+}]} &=& \int^{\infty}_{0}(x-X_{0})^{+}h_{2}(x)dx \nonumber\\
&=& \int^{\infty}_{X_{0}}{xq\theta_{1}e^{-\theta_{1}x}dx} + \int^{\infty}_{X_{0}}{x(1-q)\theta_{2}e^{-\theta_{2}x}dx}-X_{0}\int_{X_{0}}^{\infty}{h_{2}(x)dx} \nonumber\\
&=& qX_{0}e^{-\theta_{1}X_{0}} + \frac{q}{\theta_{1}}\int_{X_{0}}^{\infty}{\theta e^{-\theta_{1} x}dx} + (1-q)X_{0}e^{-\theta_{2}X_{0}} + \frac{1-q}{\theta_{2}}\int_{X_{0}}^{\infty}{\theta_{2}e^{-\theta_{2}x}dx} \nonumber\\
&& -X_{0}\bigg[1-H_{2}(X_{0}) \bigg] \nonumber
\end{eqnarray}
With the exponential probability density function $f_{\theta}(x)=\theta e^{-\theta x}$, and the exponential cumulative distribution function $F_{\theta}(x)= 1-e^{-\theta x}$, define the complementary distribution function of $F_{\theta}(x)$ as $\bar{F}_{\theta}(x)$, then the first partial moment can be written as:
\begin{eqnarray}
\mathbb{E}{[(X-X_{0})^{+}]}&=& qX_{0}e^{-\theta_{1}X_{0}} + (1-q)X_{0}e^{-\theta_{2}X_{0}} + \frac{q}{\theta_{1}}\bigg[1-F_{\theta_{1}}(X_{0})\bigg]+\frac{1-q}{\theta_{2}}\bigg[1-F_{\theta_{2}}(X_{0}) \bigg] \nonumber\\
&&-X_{0}\bigg[ 1- H_{2}(X_{0})\bigg] \nonumber
\end{eqnarray}


\section{Figures of the total downtime of the test bed}

\begin{figure}
\centering
\includegraphics[width=1.2\linewidth]{lognormal.eps}
 \caption{The simulation curve and approximation curve of the downtime density distribution under lognormal distributed failure rate}
 \label{fig:lognormal}
\end{figure}


\begin{figure}
\centering
\includegraphics[width=1.2\linewidth]{uniform.eps}
 \caption{The simulation curve and approximation curve of the downtime density distribution under uniform distributed failure rate}
  \label{fig:uniform}
\end{figure}

\end{appendices}

\newpage
\section{References}
%\bibliographystyle{model2-names}
\bibliographystyle{plainnat}
\bibliography{reference}


\end{document}  